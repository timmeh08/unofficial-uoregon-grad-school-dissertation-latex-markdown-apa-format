% -*- latex -*-

%As already mentioned in \ref{sec:mc}, three categories of background affect the analysis: events with true photon, events with fake photon from misidentified electron or jet, and event with fake $\ETmiss$. The last two contributions actually overlap, i.e. in order to pass the SR selections, some of the background processes must contain both fake photon and fake $\ETmiss$. The only process which constitutes a background fully due to fake $\ETmiss$ is $\gamma jets$ with direct photons, which is subdominant in the analysis regions and is therefore estimated purely from MC simulation.

As the vertex BDT underlies the full analysis, we look at the data/MC comparison of the vertex BDT in section~\ref{sec:vertex-VR}. This comparison is dependent on other backgrounds, which are outlined below. 

As already mentioned in \ref{sec:mc}, three categories of background affect the analysis:
\begin{itemize}
\item [(a)] events with a genuine photon and genuine $p_T^{miss}$: coming from $Z(\to\nu\nu)\gamma$ and $W(\to\ell\nu)\gamma$ processes --- in the latter, the charged lepton is lost or not identified;
\item [(b)] events with a jet or an electron misidentified as a photon --- called $j\rightsquigarrow\gamma$ and $e\rightsquigarrow\gamma$ respectively;
\item [(c)] events without genuine $p_T^{miss}$
\end{itemize}

Most of the events belonging to category (c) are also characterized by $j\rightsquigarrow\gamma$, as they arise from multijet processes, or $\gamma$-jets final states where the photon emerge from a parton shower (``fragmentation photon''). The only process with fake $p_T^{miss}$ but genuine photon is the $\gamma$-jets final state with a photon arising from the matrix element (``direct photon''): such a process is subdominant in the analysis regions and is therefore estimated purely from MC simulation. The treatment of direct and fragmentation photons may be analysis-dependent: in several inclusive photon measurements from QCD, direct and fragmentation photons are both considered together as signal. Instead, in this analysis, fragmentation photons are treated as $j\rightsquigarrow\gamma$, for the following reasons: these photons are typically embedded into a hadronic jet, and therefore their isolation profile is more similar to that of a jet rather than that of an isolated photon; moreover, the signal sought in this analysis, $H\to\gamma\gamma_d$, is characterized by an isolated photon much more similar to a direct photon than to a fragmentation photon --- we do not expect fragmentation photons from Higgs boson decays.

\begin{figure}[p]
  \begin{center}
    \includegraphics[height=0.8\textheight]{images/sketch-SR-CRs.pdf}
    \caption{Sketch of the SR and all CR and probe regions used for background estimates.}
    \label{fig:sketch-SR-CRs}
  \end{center}
\end{figure}

The strategy for the estimation of the other backgrounds with true photons ($W\gamma$ and $Z\gamma$) is described in section \ref{sec:irreducible}, while the data-driven strategy for fake photons are detailed in section \ref{sec:fakephotonintro}.

What follows is a quick guide to the rest of this section: see Figure~\ref{fig:sketch-SR-CRs} for reference.
\begin{itemize}
\item The background from $Z(\to\nu\nu)\gamma$ is modelled from MC simulation, corrected by a scaling factor $K_{Z\gamma}$ which is derived from a 2$\mu$-CR, enriched of $Z(\to\mu\mu)\gamma$ events; the scaling factor is computed as
  $\displaystyle K_{Z\gamma}=\frac{N_{data}(Z(\to\mu\mu)\gamma)}{N_{MC}(Z(\to\mu\mu)\gamma)}$, after subtracting events with fake photons from the data 2$\mu$-CR.
\item The background from $W(\to\ell\nu)\gamma$ is modelled from MC simulation, corrected by scaling factors $K_{W\gamma}$ derived from a 1$\mu$-CR, enriched of $W(\to\mu\nu)\gamma$ events; the scaling factors are computed as
  $\displaystyle K_{W\gamma}=\frac{N_{data}(W(\to\mu\nu)\gamma)}{N_{MC}(W(\to\mu\nu)\gamma)}$, after subtracting events with fake photons from the data 1$\mu$-CR.
\item The background from $j\rightsquigarrow\gamma$, both in SR and in 1$\mu$-CR and 2$\mu$-CR, is computed starting from events in non-isolated-$\gamma$-probe regions, defined with a non-isolated photon instead of the photon; each event is weighted by a fake rate
  $\displaystyle f_{j\gamma}=\frac{N_{isol}(j\rightsquigarrow\gamma)}{N_{nonisol}(j\rightsquigarrow\gamma)}$, defined as the ratio of $j\rightsquigarrow\gamma$ that pass or fail an isolation criterion. 
  To compute the $f_{j\gamma}$, one needs to know the isolation profile of $j\rightsquigarrow\gamma$ passing the Tight identification. The MC simulation is not reliable for that, therefore the isolation profile of $j\rightsquigarrow\gamma$ passing the Tight identification is inferred from a data sample of reconstructed photon objects passing Loose identification and failing the Tight identification (which are mostly $j\rightsquigarrow\gamma$), applying a transform extracted from two MC samples of $j\rightsquigarrow\gamma$, satisfying the Loose-not-Tight and the Tight identification respectively.
  This study is detailed in Section~\ref{sec:jetfake}.
\item The background from $e\rightsquigarrow\gamma$, both in SR and in 1$\mu$-CR and 2$\mu$-CR, is computed starting from events in $e$-probe regions, defined by requiring an electron instead of the photon; each event is weighted by a fake rate
  $\displaystyle f_{e\gamma}=\frac{N(e\rightsquigarrow\gamma)}{N(e\rightsquigarrow e)}$, defined as the ratio of electrons that are (wrongly) reconstructed as photons, and those that are (correctly) reconstructed as electrons.
  The $f_{e\gamma}$ is computed as ratios between event counts reconstructed as $\gamma e$ and as $ee$ with invariant mass around the $Z$-peak. Most of the $\gamma e$ events come from $Z\to ee$ with one $e\rightsquigarrow\gamma$, however a sizable contribution with genuine photon comes from $W(\to e\nu)\gamma$: such a contribution is estimated and subtracted using events $\gamma\mu$ (mostly due to $W(\to\mu\nu)\gamma$). This approach is described in Section~\ref{sec:elefake}.
\end{itemize}

\clearpage

