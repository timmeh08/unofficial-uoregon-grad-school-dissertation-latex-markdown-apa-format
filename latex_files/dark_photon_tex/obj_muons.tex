Muons are reconstructed by matching ID tracks to MS tracks or track segments, by matching ID tracks to
a calorimeter energy deposit compatible with a minimum-ionizing particle, or by identifying MS tracks
passing the loose requirement and compatible with the IP. Baseline muons in this analysis must fulfil $p_{\mathrm{T}}>10$ GeV and $|\eta|<2.7$. A \texttt{Medium} identification WP is applied, and similar selection on the longitudinal and transverse impact parameter as for electrons. 
On top of these selections, selected muons must pass the \texttt{PflowLoose\_VarRad} Isolation WP. 


\begin{table}[ht]
  \caption{Muon selection criteria.}%
  \label{tab:object:muon}
  \centering
  % \resizebox{\textwidth}{!}{
  \begin{tabular}[ht]{ll}
    \toprule
    Feature & Criterion \\
    Object quality & Bad Muon Veto \\
    %Momentum calibration & Sagitta correction [used/not used] \\
    $p_{\mathrm{T}}$ cut & $p_{\mathrm{T}}>10$ GeV \\
    $|\eta|$ cut & $< 2.7$ \\
    $d_{0}$ significance cut & 3 \\
    $z_{0}$ cut & 0.5 mm \\
    \midrule
    \midrule
    \textbf{Baseline} &\\
    \midrule
    Identification working point & \texttt{Medium} \\
    \textbf{Selected} &\\
    \midrule
    Identification working point & \texttt{Medium} \\
    Isolation working point & \texttt{PflowLoose\_VarRad}\\
    \bottomrule
  \end{tabular} 
  % }
\end{table}