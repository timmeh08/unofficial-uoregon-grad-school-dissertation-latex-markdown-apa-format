In addition to the SR, orthogonal control regions (CR) and validation regions (VR) are defined. The two CRs are employed to correct for potential mismodelling in $Z\gamma$ and $W\gamma$ MC simulations, while 3 VRs are defined to validate the data-driven estimates described in detail in section \ref{sec:fakephotonintro}. 

\subsection{Control regions}
The control regions are defined to be enhanced with $W(\mu\nu)\gamma$ and $Z(\to\mu\mu)\gamma$ events, with the latter used as a proxy for the $Z(\to\nu\nu)\gamma$ process in the SR.
To achieve this, the muon veto applied in the SR is inverted, and the $\ETmiss$ related variables are recomputed treating muons as invisible particle in the $\ETmiss$ reconstruction ($\ETmiss$(inv. $\mu$)), in order to mimic the kinematic of the SR. 
\begin{itemize}
    \item the $1\mu$ CR (designed for $W\gamma$) is obtained by applying the same selection as the SR, but requiring exactly one selected muon and using the $\ETmiss$(inv. $\mu$);
    \item the $2\mu$ CR (designed for $Z\gamma$) requires exactly 2 muons and uses the $\ETmiss$(inv. $\mu$), but applies slightly looser selection than the SR to enhance the statistics. In addition, in order to reject the $Z\to\mu\mu\gamma$ decay, the invariant mass of the $\mu\mu\gamma$ system is required to be $m_{\mu\mu\gamma}>100$ GeV. The detailed selections are summarized in table \ref{tab:region_summary}.
\end{itemize}

Figure \ref{fig:MCcr} 
shows the $\mT$ distributions in the two CRs, which are confirmed to be enhanced with the desired background process. The $\jfakey$ background contribution is determined through the data-driven method described in section \ref{sec:jetfake}

\begin{figure} [H]%[!htbp] t!
    \centering
    \includegraphics[width=0.45\textwidth]{../images/MCplots/mt_1muCR.png}
    \includegraphics[width=0.45\textwidth]{../images/MCplots/mt_2muCR.png}\\
    \caption{Background and data $m_T$ distribution in the $1\mu$ CR and $2\mu$ CR. Data-driven $\jfakey$ are included in the plot. The bottom panel shows the MC/data ratio.}
    \label{fig:MCcr}
\end{figure}

\subsection{Validation regions}
Three validation regions are defined, two for the $\jfakey$ (low $\ETmiss$ significance VR, low $\dphimety$ VR) and one for the $\efakey$ estimates (low $\mT$ VR). 
The choice of two VRs for $\jfakey$ validation is done in order to be able to cross-check the modelling both of the $\ETmiss$ significance variable and of the $\dphimety$ one. \\
\begin{itemize}
    \item The low $\ETmiss$ significance VR is defined as the SR, but requiring $2 < \ETmiss sig < 6$. The lower threshold is applied to reject a mismodelled region, which is of no interest for the analysis.
    \item The low $\dphimety$ maintains the same selection as the SR, but requiring $0.5<$\dphimety$<1$. Also in this case, the lower cut excludes a not relevant mismodelled region.
    \item The low $\mT$ region is obtained by inverting the $\mT$ cut ($80 <\mT < 90 $ GeV). This choice is determined by the fact that the main contribution to this background comes from $W(e\nu)jets$ events, where the electron is misidentified as a photon. Therefore, the $\mT$ will peak around the mass of the W boson.
\end{itemize}
The detailed study in these VRs can be found in sections \ref{sec:jetfake} and \ref{sec:elefake}.

