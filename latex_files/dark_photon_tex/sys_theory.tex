\textcolor{red}{PRELIMINARY draft }
Theory uncertainties for MC samples are estimated following the PmgSystematics recommendations, using the on-the-fly weights available in the MC samples. 

%An uncertainty of $\pm6\%$ is assigned on the inclusive \ttbar production cross-section at NNLO+NNLL which is the result of adding in quadrature contributions from renormalisation and factorisation scale variations (obtained by independently varying the parameters $\mu_R$ and $\mu_F$ by a factor $0.5$ and $2.0$ and taking the envelope), PDF and $\alpha_S$ variations (where variations follow the PDF4LHC treatment~\cite{ThePDF4LHCWorGroIntRec} with the typical \texttt{Var3c} variations of the strong coupling constant $\alpha_S$) and mass uncertainty (which follows from variations of the top mass by $\pm 1~\GeV$)~\cite{syst:ttbarxsec}. The associated systematic is labelled \texttt{ttbar\_cross\_section}.
Missing higher order contributions are estimated by adding in quadrature contributions from renormalisation and factorisation scale variations, which are obtained by independently varying the parameters $\mu_R$ and $\mu_F$ by a factor $0.5$ and $2.0$ and taking the envelope. 

Uncertainties on the choice of the PDF set used for event simulation are estimated by using the PDF4LHC and NNPDF error sets following the typical PDF4LHC recommendation~\cite{ThePDF4LHCWorGroIntRec} and taking the envelope.  The uncertainty on the strong coupling constant $\alpha_S$ is evaluated by using the same PDF set with two different choices of $\alpha_S$. In the fit, both parameters are added in quadrature and treated as correlated parameters.

The initial state radiation (ISR) modelling, i.e. the amount of predicted ISR in an event by the parton showering algorithm can be estimated by making use of the built-in up and down variations of the \texttt{Var3c} parameter. 

The amount of final state radiation (FSR) in an event is estimated by varying the factorisation scale by factors $0.5$ and $2.0$ inside \texttt{Pythia}. 

