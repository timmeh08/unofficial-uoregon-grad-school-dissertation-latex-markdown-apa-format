Photon candidates (baseline photons) are reconstructed from clustered energy deposits in the EM calorimeter. They must
fulfil $p_{\mathrm{T}}>25$ GeV, $|\eta|<2.37$, outside of the transition region between barrel and endcap calorimeters ($1.37< |\eta| < 1.52$), and \texttt{Tight} identification criteria. \\
Selected photons are additionally required to be isolated according to the \texttt{TightCaloOnly} Workin Point (WP). An additional selection on the photon $p_{\mathrm{T}}$ is applied at event level, requiring one selected photon with $p_{\mathrm{T}}>50$ GeV. 

\begin{table}[ht]
  \caption{Photon selection criteria.}%
  \label{tab:object:photon} 
  \centering
  % \resizebox{\textwidth}{!}{
  \begin{tabular}{ll}
    \toprule
    Feature & \multicolumn{1}{c}{Criterion} \\
    Object quality & Not from a bad calorimeter cluster (\texttt{BADCLUSPHOTON})\\ %\cline{2-2}
    Photon cleaning & \texttt{passOQquality} \\
    Pseudorapidity range & $(|\eta| < 1.37) \quad || \quad (1.52 < |\eta| < 2.37)$ \\
    Energy calibration & \texttt{es2022\_R22\_PRE} (ESModel)\\
    \midrule\midrule
    \textbf{Baseline photons} &\\ 
    \midrule
 %   Energy & $E > \SI[parse-numbers=false]{XX}{\GeV}$ \\
    Transverse energy & $p_{\mathrm{T}}>10$ GeV \\
    Identification & \texttt{Loose} \\
    \midrule
    \textbf{Selected photon} &\\ 
    \midrule
    Transverse energy & $p_{\mathrm{T}}>50$ GeV \\
    Identification & \texttt{Tight} \\
    Isolation &  \texttt{TightCaloOnly} \\
    \bottomrule
  \end{tabular}
  %  }
\end{table}