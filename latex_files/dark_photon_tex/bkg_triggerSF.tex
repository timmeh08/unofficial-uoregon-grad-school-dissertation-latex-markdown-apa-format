
The trigger used in the analysis is a combination of requirements on trigger-level cuts on $p_T^\gamma,~p_T^{miss},~m_T$, that correspond to analogous cuts on the offline variables, according to Table~\ref{tab:trig_offline_cuts}.

\begin{table}[h]
  \begin{center}
    \begin{tabular}{c|cc}
      & trigger & offline \\
      \hline
      $p_T^\gamma$ & $>50$~GeV & $>50$~GeV \\
      $p_T^{miss}$ & $>70$~GeV & $>100$~GeV \\
      $m_T$        & $>80$~GeV & $>90$~GeV \\
    \end{tabular}
  \end{center}
  \caption{
    Thresholds on trigger variables and offline variables.
  }
  \label{tab:trig_offline_cuts}
\end{table}

As a result, the efficiency of the trigger selection with respect to the offline selection, defined as
\begin{equation}
  \epsilon(\textrm{trigger}|\textrm{offline}) = \frac{ N(\textrm{trigger}~\&\&~\textrm{offline}) }{ N(\textrm{offline}) }
  \label{eq:trig_eff_wrt_offline}
\end{equation}
is lower than 100\% and needs to be evaluated. Figure~\ref{fig:trig_eff_MC} display the efficiency curves from MC, for the signal $H\to\gamma\gamma_d$ and for two background processes with $p_T^{miss}$, i.e. $Z(\to\nu\nu)\gamma$
and $W(\to\mu\nu)\gamma$. As the muon is not consideret for trigger-level $p_T^{miss}$ evaluation, also the offline $p_T^{miss}$ and $m_T$ are computed treating the muon as invisible.

\begin{figure}[!htbp]
  \begin{center}
    \includegraphics[width=0.48\columnwidth]{images/trigger_eff_MC_vs_pTy_pTy050_pTmiss100_mT090_allSigMC.pdf}
    \includegraphics[width=0.48\columnwidth]{images/trigger_eff_MC_vs_pTmiss_pTy050_pTmiss100_mT090_allSigMC.pdf}

    \includegraphics[width=0.48\columnwidth]{images/trigger_eff_MC_vs_mT_pTy050_pTmiss100_mT090_allSigMC.pdf}
  \end{center}
  \caption{
    Efficiency curves of $\epsilon(\textrm{trigger}|\textrm{offline})$ as functions of $p_T^\gamma,~p_T^{miss},~m_T$, derived from MC. The vertical green lines display the position of the offline cuts: in each figure, the offline cuts
    are applied except that on the displayed variable. Besides the trigger efficiency on the signal $H\to\gamma\gamma_d$ (red plot), the analogous for the $Z(\to\nu\nu)\gamma$ (orange) and for the $W(\to\mu\nu)\gamma$ (purple) processes are displayed
    for comparison. The latter is obtained for final states containing a photon and a muon, which is treated as invisible in the evaluation of $p_T^{miss}$ and $m_T$.
  }
  \label{fig:trig_eff_MC}
\end{figure}

The efficiency curves for $Z(\to\nu\nu)\gamma$ and $W(\to\mu\nu)\gamma$ are very similar --- confirming that the trigger is insensitive to the presence of the muon. However the efficiency curves for the signal exhibit large differences. The reason has been tracked down to the different kinematics of the signal, where the value of $m_T$ is constrained in an interval $[90;150]~\mathrm{GeV}$. To prove that this is indeed the reason, the efficiency curves have been extracted again only for events with $m_T<150~\mathrm{GeV}$: the result is visible in Figure~\ref{fig:trig_eff_MC_mT150}. Indeed, now the efficiency curves for $H\to\gamma\gamma_d$ and $Z(\to\nu\nu)\gamma$ do match.

\begin{figure}[!htbp]
  \begin{center}
    \includegraphics[width=0.48\columnwidth]{images/trigger_mTbelow150_eff_MC_vs_pTy_pTy050_pTmiss100_mT090_allSigMC.pdf}
    \includegraphics[width=0.48\columnwidth]{images/trigger_mTbelow150_eff_MC_vs_pTmiss_pTy050_pTmiss100_mT090_allSigMC.pdf}

    \includegraphics[width=0.48\columnwidth]{images/trigger_mTbelow150_eff_MC_vs_mT_pTy050_pTmiss100_mT090_allSigMC.pdf}
  \end{center}
  \caption{
    Efficiency curves of $\epsilon(\textrm{trigger}|\textrm{offline})$ as functions of $p_T^\gamma,~p_T^{miss},~m_T$, derived from MC, for events with $m_T<150~\mathrm{GeV}$. The vertical green lines display the position of the offline cuts: in each figure, the offline cuts
    are applied except that on the displayed variable. Besides the trigger efficiency on the signal $H\to\gamma\gamma_d$ (red plot), the analogous for the $Z(\to\nu\nu)\gamma$ (orange) and for the $W(\to\mu\nu)\gamma$ (purple) processes are displayed
    for comparison. The latter is obtained for final states containing a photon and a muon, which is treated as invisible in the evaluation of $p_T^{miss}$ and $m_T$.
  }
  \label{fig:trig_eff_MC_mT150}
\end{figure}

To assess the accuracy of the trigger simulation, the efficiency curves from MC must be compared to those derived by data. To do so, a baseline muon trigger is used, and the used final state contains a photon and a muon, which is treated as invisible in the evaluation of $p_T^{miss}$ and $m_T$. The efficiencies to be compared are defined as:
\begin{equation}
  \epsilon(\textrm{trigger}~|~\textrm{offline}~\&\&~\mu\textrm{-trigger}) = \frac{ N(\textrm{trigger}~\&\&~\textrm{offline}~\&\&~\mu\textrm{-trigger}) }{ N(\textrm{offline}~\&\&~\mu\textrm{-trigger}) }
  \label{eq:trig_eff_wrt_offline_basetrigger}
\end{equation}
Data must be compared to the total background from MC simulation, as shown in Figure~\ref{fig:trig_eff_data}: there is good agreement between data and total background, while the agreement with $Z(\to\nu\nu)\gamma$ and $W(\to\mu\nu)\gamma$ processes is worse.

\begin{figure}[!htbp]
  \begin{center}
    \includegraphics[width=0.48\columnwidth]{images/trigger_eff_data_vs_pTy_pTy050_pTmiss100_mT090_allSigMC.pdf}
    \includegraphics[width=0.48\columnwidth]{images/trigger_eff_data_vs_pTmiss_pTy050_pTmiss100_mT090_allSigMC.pdf}

    \includegraphics[width=0.48\columnwidth]{images/trigger_eff_data_vs_mT_pTy050_pTmiss100_mT090_allSigMC.pdf}
  \end{center}
  \caption{
    Efficiency curves of $\epsilon(\textrm{trigger}|\textrm{offline}~\&\&~\mu\textrm{-trigger})$ as functions of $p_T^\gamma,~p_T^{miss},~m_T$, from data (solid dots) and MC (grey shaded histograms). The vertical green lines display the position of the offline cuts: in each figure, the offline cuts
    are applied except that on the displayed variable. Efficiency curves for the $Z(\to\nu\nu)\gamma$ (orange) and for the $W(\to\mu\nu)\gamma$ (purple) processes are also displayed
    for comparison. In all cases, the final states containing a photon and a muon, which is treated as invisible in the evaluation of $p_T^{miss}$ and $m_T$.
  }
  \label{fig:trig_eff_data}
\end{figure}

The trigger scale factors are evaluated as a ratio between the trigger efficiency in data and in MC. This can be done as a function of any of the $p_T^\gamma,~p_T^{miss},~m_T$ variables, as displayed in Figure~\ref{fig:trig_scf}.
In each plot, the bottom panel shows the scale factor as a function fo the kinematic variable; overlayed is the kinematic spectrum of the signal; the average scale factor is computed as a weighted average of the binned scale factors, accounting for the kinematic spectrum of the signal, according to:
\begin{equation}
  \langle\mathrm{scf}\rangle = \sum_{i} \frac{n_s[i]}{n_s^{tot}}\mathrm{scf}[i]
  \label{eq:avg-trig-scf}
\end{equation}
where $\mathrm{scf}[i]$ is the scale factor computed for the $i$-th bin, $n_s[i]$ is the number of signal events in that bin, and $n_s^{tot}=\sum_i n_s[i]$.

\begin{figure}[!htbp]
  \begin{center}
    \includegraphics[width=0.48\columnwidth]{images/trigger_scf_vs_pTy_pTy050_pTmiss100_mT090_finalstate_ymu_allSigMC.pdf}
    \includegraphics[width=0.48\columnwidth]{images/trigger_scf_vs_pTmiss_pTy050_pTmiss100_mT090_finalstate_ymu_allSigMC.pdf}

    \includegraphics[width=0.48\columnwidth]{images/trigger_scf_vs_mT_pTy050_pTmiss100_mT090_finalstate_ymu_allSigMC.pdf}
  \caption{
    Efficiency curves for $\epsilon(\textrm{trigger}|\textrm{offline}~\&\&~\mu\textrm{-trigger})$ (top panels) and trigger scale factors (bottom panels), as functions of $p_T^\gamma,~p_T^{miss},~m_T$. The top panels display the efficiency curves for data (solid dots) and MC total background (grey shaded histograms). Events are selected with a final state containing a photon and a muon, which is treated as invisible in the evaluation of $p_T^{miss}$ and $m_T$.
    The bottom panels also display the expected kinematic spectrum of the signal (hatched red histograms). The average scale factors are also displayed,
  }
  \label{fig:trig_scf}
  \end{center}
\end{figure}

The scale factors as functions of $p_T^\gamma$ and $m_T$ appear somewhat ``jumpy'': this is due to fluctuations in the efficiency curve of the MC background, as it is evident from the figures. On the opposite, the scale factors as functions of $p_T^{miss}$ appear smoother, and compatible with 1: their values are displayed in Table~\ref{tab:trig_scf_vs_pTmiss}. The averaged scale factor is:
\begin{equation}
  \langle\mathrm{scf}\rangle_{p_T^{miss}} = 1.015 \pm 0.015
  \label{eq:trig_avg_scf_vs_pTmiss}
\end{equation}
which will be used as a correction factor to the signal selection efficiency.

\begin{table}[!htbp]
  \begin{center}
    \begin{tabular}{c|c}
      $p_T^{miss}~\in$ & scf \\
      \hline
      $[100;125]$ & $1.020\pm0.010$ \\
      $[125;150]$ & $1.016\pm0.009$ \\
      $[150;200]$ & $1.003\pm0.008$ \\
      $[200;300]$ & $0.993\pm0.009$ 
    \end{tabular}
    \caption{
      Trigger scale factors evaluated in bins of $p_T^{miss}$.
    }
    \label{tab:trig_scf_vs_pTmiss}
  \end{center}
\end{table}

\clearpage
