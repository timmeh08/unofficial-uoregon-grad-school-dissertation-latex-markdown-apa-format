A statistical combination of the Run 3 $\Htoyyd$ search (Run-3 ggF channel) conducted in this study with the Run 2 searches for $\Htoyyd$ in $\gamma + \ETmiss + \text{VBF jets}$ (Run-2 VBF channel) and $\gamma + \ETmiss + Z(\to\ell\ell, \ell = e,\mu)$ (Run-2 $ZH$ channel) final states is performed for the SM Higgs boson scenario.
The results of the Run-3 ggF channel are obtained from the simultaneous fit described in Section~\ref{sec:simfit}, {\color{red}implementing the data-driven estimations of the $\jfakey$ (Section~\ref{sec:jetfake}) and $\efakey$ (Section~\ref{sec:elefake}) backgrounds with their systematic uncertainties, while applying the flat systematic uncertainty of 10\% to all the other MC samples.}
The fit is performed using a so-called ``mixed-Asimov'' dataset, where data are unblinded in the $1\mu$ and $2\mu$ CRs, and ``modified pseudo-data'' -- extrapolated from a background-only fit to these CRs using real data -- are used in the SR.
The Run-2 VBF channel results are taken from the analysis reported in Ref.~\cite{EXOT-2021-17}, whereas the results for the Run-2 $ZH$ channel are derived from Ref.~\cite{HDBS-2019-13}.
The combination of these two searches was already done by the ATLAS Collaboration, resulting in an observed (expected) 95\% CL upper limit on the branching ratio $\mathcal{B}(\SMHtoyyd)$ of $1.3\%(1.5\%)$~\cite{ATLAS:2024cju}.

\subsection{Overlap check}
\label{sec:overlap_check}
In order to prevent potential double-counting, which could introduce unintended biases in the combination, it is essential to ensure that the three input channels have no overlapping events or, at most, a negligible level of overlap.
Naturally, due to the statistical independence of the Run-3 and Run-2 datasets, the Run-3 ggF channel is mutually exclusive with respect to both the Run-2 VBF and $ZH$ channels.
While the orthogonality of the fit regions defined in the Run-2 analyses has already been verified in Ref.~\cite{ATLAS:2024cju}, following the procedure outlined in Ref.~\cite{overlapprocedure}.
This validation was performed using either the Full Run-2 data or the MC signal sample of the SM Higgs boson production via the VBF mode.
In all cases, no event overlap was observed between the Run-2 VBF and $ZH$ channels.

\subsection{Statistical method}
\label{sec:stat_method}
The combination presented in this section relies on a combined likelihood function defined as the product of the likelihoods from the individual channels entering the combination, which are themselves products of likelihoods computed from the final observables in various categories in a single analysis.

The common parameter of interest (POI) targeted in this combination is the branching ratio of $\Htoyyd$, denoted by $\mu$.
Systematic uncertainties are included as nuisance parameters, denoted by $\bm{\theta}$, and are constrained by Gaussian probability density functions.
These encode information from \textit{auxiliary} measurements and measure the effect of systematic uncertainties.
The likelihood also includes normalisation factors, denoted by $\bm{\kappa}$, which are floated in the fit without constraints to adjust the agreement with data of background components in their corresponding CR(s).

The combined likelihood can be written as~\cite{Cranmer:1456844}
\begin{equation}
    \mathcal{L}\left(\mathrm{data}\vert\mu,\bm{\kappa},\bm{\theta}\right) = \prod_{c=1}^{N_\mathrm{cats}}\mathcal{L}_c\left(\mathrm{data}\vert\mu,\bm{\kappa},\bm{\theta}\right)\prod_{k=1}^{N_\mathrm{cons}}\mathcal{G}\left(\tilde{\theta}_k\vert\theta_k\right),
    \label{eq:combined_likelihood}
\end{equation}
where $N_\mathrm{cats}$ is the number of categories, $N_\mathrm{cons}$ is the number of constrained NPs, $\tilde{\theta}_k$ is the global observable corresponding to $\theta_k$, $c$ is the index for the event categories, $k$ is the index for the constrained NPs, and $\mathcal{G}$ represents a Gaussian distribution.

The likelihood for a given category $\mathcal{L}_c$ can be expressed as the product of Poisson distributions, one for each bin in the histogram of the observable considered for that category, as
\begin{equation}
    \mathcal{L}_c\left(\mathrm{data}\vert\mu,\bm{\kappa},\bm{\theta}\right) = \prod_{i=1}^{N_c^\mathrm{bins}}\mathcal{P}\left(n^i_c\middle\vert\sum_{S} S^i_{c}(\mu,\bm{\theta}) + \sum_{B}B^i_{c}(\bm{\kappa},\bm{\theta}) \right),
    \label{eq:cat_likelihood}
\end{equation}
where $N^\mathrm{bins}_c$ is the number of bins, $i$ is the bin index, $\mathcal{P}$ denotes a Poisson distribution, $n^i_c$ is the number of (observed or Asimov) data events in each bin, $S^i_{c}$ and $B^i_{c}$ are the expected yields of a given signal sample $S$ and background sample $B$, respectively, in each bin.

The $\mathrm{CL_s}$ frequentist method~\cite{CLsmethod} following the asymptotic formulae~\cite{Cowan_2011} is used to derive the 95\% CL upper limit on the POI with the profile likelihood ratio test statistic $(q_\mu)$, defined as
\begin{equation}
    q_\mu = -2\ln\frac{\mathcal{L}\left(\mu,\hat{\hat{\bm{\kappa}}}(\mu), \hat{\hat{\bm{\theta}}}(\mu)\right)}{\mathcal{L}\left(\hat{\mu},{\hat{\bm{\kappa}}}, {\hat{\bm{\theta}}}\right)},
    \label{eq:qmu}
\end{equation}
where the numerator indicates the values of $\bm{\kappa}$ and $\bm{\theta}$ that maximise $\mathcal{L}$ for a given value of $\mu$, and the denominator is evaluated for the values $\hat{\mu}$, ${\hat{\bm{\kappa}}}$, ${\hat{\bm{\theta}}}$ which which jointly maximise the likelihood.

While the original POIs of the Run-3 ggF and Run-2 VBF channels are already $\mathcal{B}(\Htoyyd)$, the original Run-2 $ZH$ channel POI is defined as the signal strength assuming $\mathcal{B}(\Htoyyd) = 5\%$.
It is therefore re-scaled by a factor of 0.05 to be correlated with the other channels' POIs in the combination.
It should also be noted that the \textbf{current} combination results, presented below, are obtained using fully unblinded data in the Run-2 channels, while the SR of the Run-3 ggF channel remains blinded, and the aforementioned ``mixed-Asimov'' dataset is used for this channel.

\subsection{Correlation scheme of systematic uncertainties}
\label{sec:correlation_scheme}
All systematic uncertainties considered in the input channels are included in the combination.
The Run-3 ggF channel systematic uncertainties, including those related to the data-driven fake-photon background estimates, the trigger, and the flat uncertainty of 10\% are treated as uncorrelated in the combination. 
The correlation of systematic uncertainties between the Run-2 VBF and $ZH$ channels follows a scheme similar to that described in Ref.~\cite{ATLAS:2024cju}, as follows,
\begin{itemize}
    \item Uncertainties related to the luminosity and pile-up modelling are correlated. 
    \item Uncertainties related to EG, muon, and $\ETmiss$ are correlated, except for those that are heavily pulled or constrained in either of the original input channels.
    \item JES uncertainties are correlated except the pulled/over-constrained ones in the Run-2 VBF channel, while JER uncertainties are fully uncorrelated.
    \item Uncertainties related to signal and background modelling are uncorrelated.
\end{itemize}
In addition, any uncertainties, which are over-constrained $(< 70\%)$ or pulled $(> 50\%)$ in the Run-2 VBF-$ZH$ combined fit to data, are also not correlated in this combination.

\subsection{Preliminary results}
\label{sec:combinationWithPublishedRun2_results}
Table~\ref{tab:comb_massless} presents the preliminary 95\% CL upper limits on $\mathcal{B}(\SMHtoyyd)$, derived from the combination of the Run-3 ggF channel (the search conducted in this study) with the already published Run-2 VBF and $ZH$ channels.
The statistical procedure follows the strategy outlined in Section~\ref{sec:stat_method}, and the treatment of uncertainty correlation is based on the scheme described in Section~\ref{sec:correlation_scheme}.
In addition, the individual limits from each channel are also provided. 
All the results are obtained using the combination framework\footnote{https://gitlab.cern.ch/zhangruiPhysics/DarkPhotonCombination} previously employed in the Run-2 combined search reported in Ref.~\cite{ATLAS:2024cju}.
The Run-2 limits are reproduced using fully unblinded data and are consistent with the ``reference'' results in the respective publications. 
In contrast, the combined limits and those from the Run-3 ggF channel are computed with the SR of the Run-3 ggF analysis remaining blinded, and the “mixed-Asimov” dataset is used instead.

\begin{table}[!htbp]
    \centering
    \resizebox{\columnwidth}{!}{
	    \begin{tabular}{ l|
            S[round-mode=figures, round-precision=3]
            S[round-mode=figures, round-precision=3]
            S[round-mode=figures, round-precision=3]
            S[round-mode=figures, round-precision=3]
            S[round-mode=figures, round-precision=3]
            S[round-mode=figures, round-precision=3]
            }
            \toprule
                        & {Obs./mix.Asim.}     & {Exp.}     & {$-2\sigma$}  & {$-1\sigma$}  & {$+1\sigma$}  & {$+2\sigma$} \\
            \midrule
            Run-2 VBF Reference~\cite{EXOT-2021-17}    & 0.018	   & 0.017    &             & 0.012       & 0.024       &          \\
            Run-2 VBF Reproduced & 0.018441 & 0.017271 & 0.009270 & 0.012445 & 0.024980 & 0.034412 \\
            \midrule
            Run-2 $ZH$ Reference~\cite{HDBS-2019-13} & 0.02278    & 0.0282     & 0.01458       & 0.01976       & 0.04151       & 0.0606     \\
            Run-2 $ZH$ Reproduced  & 0.022766 & 0.028211 & 0.015142 & 0.020328 & 0.039938 & 0.057167 \\
            \midrule
            Run-3 ggF & 0.019559 & 0.019585 & 0.010512 & 0.014112 & 0.027515 & 0.037947 \\
            \midrule
            Run2+3 combined  & 0.010709 & 0.011668 & 0.006262 & 0.008407 & 0.016525 & 0.022082 \\
            \bottomrule
	    \end{tabular}
	}
    \caption{
        Exclusion upper limits at 95\% CL on $\mathcal{B}(\SMHtoyyd)$ from the individual Run-3 ggF, Run-2 VBF and $ZH$ channels as well as their combination.
        The Run-3 ggF and combination limits are obtained using the ``mixed-Asimov'' data, while the other Run-2 limits are reproduced in this combination study using fully unblinded data, and are verified with ``Reference'' results taken from the individual publications.
    }
	\label{tab:comb_massless}
\end{table}

The limits of this combination and the input channels are also represented in Figure~\ref{fig:combinedlimit_mu}.

\begin{figure}[htbp]
    \centering
    \includegraphics[width=.6\textwidth]{../images/combination_with_Run2/limits_mu_SIG.pdf}
    \caption{
        95\% CL upper limits on $\mathcal{B}(\SMHtoyyd)$ from the individual Run-3 ggF, Run-2 VBF and $ZH$ channels as well as their combination.
        The Run-3 ggF and combination limits are obtained using the ``mixed-Asimov'' data, while the other Run-2 limits are reproduced in this combination study using fully unblinded data.
    }
    \label{fig:combinedlimit_mu}
\end{figure}

The expected and observed or ``mixed-Asimov'' values of the profile likelihood ratio test statistic, defined in Equation~\ref{eq:qmu}, as a function of the POI $\mu \equiv \mathcal{B}(\SMHtoyyd)$ are illustrated for the individual analyses and their combination in Figure~\ref{fig:comb_likelihood_scan}.

\begin{figure}[htbp]
	\centering
	\subfloat[]{\includegraphics[scale=0.3]{../images/combination_with_Run2/likelihoods_mu_SIG_expected.pdf}\label{fig:exp_likelihood_scan}}
	\subfloat[]{\includegraphics[scale=0.3]{../images/combination_with_Run2/likelihoods_mu_SIG_observed.pdf}\label{fig:obs_likelihood_scan}}
	\caption{
        \ref{fig:exp_likelihood_scan} Expected and \ref{fig:obs_likelihood_scan} observed / ``mixed-Asimov'' value of $-2\ln\Lambda$, where $\Lambda$ denotes the profile likelihood ratio, as a function of $\mu \equiv \mathcal{B}(\SMHtoyyd)$ for the individual channels and their combination.}
	\label{fig:comb_likelihood_scan}
\end{figure}

Figure~\ref{fig:combined_corrmatrix} shows the correlation matrix of the POI $\mu$, normalisation factors, and constrained NPs, obtained from the best fit to ``mixed-Asimov'' data for the combination.

\begin{figure}[t]
    \centering
    \includegraphics[width=1.\textwidth]{../images/combination_with_Run2/correlation.pdf}
    \caption{
        Correlation of the POI $\mu$, normalisation factors, and constrained NPs in the combination from the best fit to ``mixed-Asimov'' data. Parameters with all values less than 0.2 are pruned.
    }
    \label{fig:combined_corrmatrix}
\end{figure}

Top 20 constrained NPs having the highest impacts on the POI $\mu$ in the combined fit to ``mixed-Asimov'' data together with their post-fit pulls are represented in Figure~\ref{fig:combined_ranking20}.

\begin{figure}[htbp]
    \centering
    \includegraphics[width=.9\textwidth]{../images/combination_with_Run2/ranking_rank_0001_to_0020.pdf}
    \caption{
        Top 20 ranking constrained NPs and their post-fit pulls for the combination, obtained using ``mixed-Asimov'' data.
        NPs related to the statistical uncertainties are represented with ``gamma'' prefix and centered at 1 but shifted to 0 for visualisation purpose.
    }
    \label{fig:combined_ranking20}
\end{figure}