At first, a background-only fit is performed simultaneously on data in the CR. The effect of the pruning on the NPs is summarised in figure \ref{fig:pruning}. 
Figures \ref{fig:pie_pre_syst} and \ref{fig:pie_post_syst} show the pre-fit and post-fit background composition in the CRs and in the SR, and the corresponding yields are summarized in tables \ref{tab:yields_pre_syst} and \ref{tab:yields_post_syst}, while the plots in figure \ref{fig:regions_jy_ey_Fit_syst} show the pre- and post-fit $m_T$ distributions in SR, 1$\mu$CR and 2$\mu$CR. The comparison between data and background expectations after the fit shows a perfect agreement: 
the binned $k_W$ results in a post-fit data/background ratio exactly flat and equal to 1, while an inclusive $k_Z$ provides a less flat ratio, but still equal to 1 within 1$\sigma$. The values of the k-factors and their related statistical uncertainties are listed in table \ref{tab:kfactors_syst}.\\
\begin{figure}[ht!]
\centering
\includegraphics[width=0.5\linewidth]{images/syst/bonly/Pruning.pdf}
\label{fig:pruning}
\caption{Summary of the effect of the pruning applied to the nuisance parameters.}
\end{figure}
\begin{figure}[ht!]
\centering
    \begin{minipage}{0.45\textwidth}
    \centering
    \includegraphics[width=\linewidth]{images/syst/PieChart.pdf}
    \caption*{(a) Pre-fit pie charts.}
    \label{fig:pie_pre_syst}
    \end{minipage}
\quad
    \begin{minipage}{0.45\textwidth}
    \centering
    \includegraphics[width=\linewidth]{images/syst/PieChart_postFit.pdf}
    \caption*{(b) Post-fit pie charts.}
    \label{fig:pie_post_syst}
    \end{minipage}

\caption{Pre-fit and post-fit SR and CRs composition.}
\label{fig:piecharts_syst}

\end{figure}

\vspace{1 cm}


\begin{table}[ht!]
    \centering
    \begin{tabular}{|c|c|c|c|}
    \hline
       \textbf{Process}  & \textbf{SR} & \textbf{1$\mu$CR} & \textbf{2$\mu$CR} \\
    \hline
   $W\gamma$   & 484.958 $\pm$ 45.1606 & 754.479 $\pm$ 44.2464 & 0.199621 $\pm$ 0.198648 \\ 
    \hline
  j$\rightarrow\gamma$   & 722.62 $\pm$ 123.54 & 265.19 $\pm$ 45.67 & 53.22 $\pm$ 9.46 \\ 
    \hline
  $e\rightsquigarrow\gamma$   & 944.228 $\pm$ 92.5933 & 73.8749 $\pm$ 7.32793 & 4e-06 $\pm$ 0 \\ 
    \hline
  $\gamma$j (direct)   & 82.5367 $\pm$ 1.83165 & 0.0381443 $\pm$ 0.000779153 & 3.93752e-06 $\pm$ 6.23857e-08 \\ 
    \hline
  $Z\gamma$   & 268.89 $\pm$ 21.0022 & 22.009 $\pm$ 1.99882 & 85.7296 $\pm$ 6.35537 \\ 
\hline 
  Total  & 2647.93 $\pm$ 168.149 & 1115.93 $\pm$ 65.3799 & 139.149 $\pm$ 11.4758 \\ 
\hline 
  data   & blind & 950 & 123 \\ 
    \hline
    
    \end{tabular}
    \caption{Pre-fit yields.}
    \label{tab:yields_pre_syst}
\end{table}


\begin{table}[ht!]
    \centering
    \begin{tabular}{|c|c|c|c|}
    \hline
       \textbf{Process}  & \textbf{SR} & \textbf{1$\mu$CR} & \textbf{2$\mu$CR} \\
    \hline
    
  $W\gamma$   & 381.485 $\pm$ 87.2048 & 587.971 $\pm$ 47.7165 & 0.152542 $\pm$ 0.137702 \\
    \hline
  j$\rightarrow\gamma$   & 737.687 $\pm$ 106.085 & 270.76 $\pm$ 39.2197 & 54.3738 $\pm$ 8.12622 \\ 
    \hline
  $e\rightsquigarrow\gamma$   & 944.163 $\pm$ 52.7012 & 73.8698 $\pm$ 4.16823 & 4e-06 $\pm$ 0 \\ 
    \hline
  $\gamma$j (direct)   & 82.5371 $\pm$ 1.76759 & 0.0381486 $\pm$ 0.00077822 & 7.94093e-06 $\pm$ 1.24263e-07 \\ 
    \hline
  $Z\gamma$   & 212.137 $\pm$ 49.7763 & 17.3597 $\pm$ 3.55907 & 67.7397 $\pm$ 13.4138 \\ 
    \hline
\hline 
  Total  & 2358.1 $\pm$ 70.7407 & 949.999 $\pm$ 30.6804 & 122.266 $\pm$ 10.7207 \\ 
\hline 
  data   & blind & 950 & 123 \\ 
    \hline
    
    \end{tabular}
    \caption{Post-fit yields.}
    \label{tab:yields_post_syst}
\end{table}

\begin{figure}[ht!]
\centering
    \begin{minipage}{0.3\textwidth}
    \centering
    \includegraphics[width=\linewidth]{images/syst/bonly/SR.pdf}
    \caption*{(a) Pre-fit SR.}
    \label{fig:SR_syst}
    \end{minipage}
\quad
    \begin{minipage}{0.3\textwidth}
    \centering
    \includegraphics[width=\linewidth]{images/syst/bonly/OneMuCR.pdf}
    \caption*{(b) Pre-fit 1$\mu$CR.}
    \label{fig:1mu_syst}
    \end{minipage}
\quad
    \begin{minipage}{0.3\textwidth}
    \centering
    \includegraphics[width=\linewidth]{images/syst/bonly/TwoMuCR.pdf}
    \caption*{(c) Pre-fit 2$\mu$CR.}
    \label{fig:2mu_syst}
    \end{minipage}


    \begin{minipage}{0.3\textwidth}
    \centering
    \includegraphics[width=\linewidth]{images/syst/bonly/SR_postFit.pdf}
    \caption*{(d) Post-fit SR.}
    \label{fig:SR_post_syst}
    \end{minipage}
\quad
    \begin{minipage}{0.3\textwidth}
    \centering
    \includegraphics[width=\linewidth]{images/syst/bonly/OneMuCR_postFit.pdf}
    \caption*{(e) Post-fit 1$\mu$CR.}
    \label{fig:1mu_post_syst}
    \end{minipage}
\quad
    \begin{minipage}{0.3\textwidth}
    \centering
    \includegraphics[width=\linewidth]{images/syst/bonly/TwoMuCR_postFit.pdf}
    \caption*{(f) Post-fit 2$\mu$CR.}
    \label{fig:2mu_post_syst}
    \end{minipage}

\caption{Pre-fit and post-fit distributions in SR and CRs.}
\label{fig:regions_jy_ey_Fit_syst}
    
\end{figure}

\vspace{5cm}
\noindent
\begin{table}[ht!]
    \centering
    \begin{tabular}{|c|c|c|c|}
    \hline
      \textbf{$m_T$ bin [GeV]} & \textbf{$k_W$} & \textbf{Error up} & \textbf{Error down} \\
    \hline
      80-110 & 0.711842 &+0.118922 &-0.109688  \\
    \hline
      110-140 & 0.691198 & +0.0959428 & -0.0898622 \\
    \hline
      140-200   & 0.790056 & +0.110348 & -0.101732 \\
    \hline
     200-300 & 1.09979 & +0.158941 & -0.14003\\
    \hline
     \textbf{$m_T$ bin [GeV]} & \textbf{$k_Z$} & \textbf{Error up} & \textbf{Error down}\\
    \hline
     Inclusive & 0.79217 & +0.177357 & -0.162185 \\
    \hline
    \end{tabular}
    \caption{Normalization factors.}
    \label{tab:kfactors_syst}
\end{table}


Figures \ref{fig:pullplot_syst} and \ref{fig:corr_syst} display the pull plot and the correlation matrix, respectively, referred to the nuisance parameters involved in the fit. No relevant pull or under/over-constraint is observed, with the exception of the \texttt{JET\_JERUnc\_Noise} NP (\textcolor{red}{to be investigated}) which also displays non-negligible correlation with the $k_Z$ normalization factor. The other nuisance parameters associated to MC experimental uncertainties do not show unexpected correlations with other systematics (i.e. they do not appear in the correlation matrix, where a threshold of 20\% is set). The large correlations between data-driven $\jfakey$ and the shape factors can be related to the contribution of $\jfakey$ in the 1$\mu$ CR.

\begin{figure}[ht!]
    \centering
    \includegraphics[width=0.5\linewidth]{images/syst/bonly/NuisPar.pdf}
    \caption{Nuisance parameters pull plot: \texttt{ff\_jy} and \texttt{ff\_ey} are the systematic uncertainties on the data-driven $j \rightsquigarrow \gamma$ and $e \rightsquigarrow \gamma$ fake rates, while \texttt{fey\_syst} is the systematic on the trigger scale factor for $e \rightsquigarrow \gamma$ estimation.\\
  .}
    \label{fig:pullplot_syst}
\end{figure}

\begin{figure}[h!]
    \centering
    \includegraphics[width=0.7\linewidth]{images/syst/bonly/CorrMatrix.pdf}
    \caption{Correlation matrix: \texttt{ff\_jy} and \texttt{ff\_ey} are the systematic uncertainties on the data-driven $j \rightsquigarrow \gamma$ and $e \rightsquigarrow \gamma$ fake rates, while \texttt{fey\_syst} is the systematic on the trigger scale factor for $e \rightsquigarrow \gamma$ estimation. Only the systematics with an effect bigger than the selected threshold (0.2) are shown. }
    \label{fig:corr_syst}
\end{figure}
\clearpage

\section{Signal-plus-background fit}
As a second step, a signal-plus-background fit is performed simultaneously in the CRs and in the SRs, using a background only asimov dataset obtained from the background-only fit results. The signal strength is, as expected, fitted to a value perfectly consistent with 0 ($0.0\pm 0.01$), and the background yields and $m_T$ distributions, are perfectly consistent with the ones obtained in the background-only fit. The behavior of NPs, once the experimental uncertainties on the signal also enter the fit, are investigated in this section. The pruning summary is reported in figure \ref{fig:splusb_pruning}, while the pull plot and correlation matrix for the nuisance parameters are shown in figures \ref{fig:pullplot_splusb} and \ref{fig:corr_splusb}, respectively. No relevant deviation with respect to the background only fit is observed.
In addition, the impact of each nuisance parameter on the fit signal strength is summarised in the ranking plot in figure \ref{fig:rankingplot}. It is obtained by fixing the value of each NP at $\pm 1\sigma$ (pre- and pos-fit) and performing a new fit with the other NPs and the POI free to float. 
\begin{figure}[ht!]
\centering
\includegraphics[width=0.5\linewidth]{images/syst/splusb/Pruning.pdf}
\label{fig:splusb_pruning}
\end{figure}

\begin{figure}[ht!]
    \centering
    \includegraphics[width=0.7\linewidth]{images/syst/splusb/NuisPar.pdf}
    \caption{Nuisance parameters pull plot for the signal-plus-background fit: \texttt{ff\_jy} and \texttt{ff\_ey} are the systematic uncertainties on the data-driven $j \rightsquigarrow \gamma$ and $e \rightsquigarrow \gamma$ fake rates, while \texttt{fey\_syst} is the systematic on the trigger scale factor for $e \rightsquigarrow \gamma$ estimation.\\}
    \label{fig:pullplot_splusb}
\end{figure}

\begin{figure}[h!]
    \centering
    \includegraphics[width=0.7\linewidth]{images/syst/splusb/CorrMatrix.pdf}
    \caption{Correlation matrix for the signal-plus-background fit: \texttt{ff\_jy} and \texttt{ff\_ey} are the systematic uncertainties on the data-driven $j \rightsquigarrow \gamma$ and $e \rightsquigarrow \gamma$ fake rates, while \texttt{fey\_syst} is the systematic on the trigger scale factor for $e \rightsquigarrow \gamma$ estimation. Only the systematics with an effect bigger than the selected threshold (0.2) are shown. }
    \label{fig:corr_splusb}
\end{figure}

\begin{figure}[h!]
    \centering
    \includegraphics[width=0.5\linewidth]{images/syst/splusb/Ranking_BRHyyd.pdf}
    \caption{Ranked impacts of the different sources of uncertainty (systematics, statistical uncertainties, normalization and shape factors) on the fit signal strength.}
    \label{fig:rankingplot}
\end{figure}


\noindent
The expected upper limit on $BR(H\rightarrow \gamma \gamma_d)$ is obtained through a limit fit with the same configuration. The obtained upper limit is 0.020377 (see table \ref{tab:limit_syst}):\\

\begin{table}[ht!]
    \centering
    \begin{tabular}{|c|c|c|c|c|c|}
    \hline
       & \textbf{-2$\sigma$} & \textbf{-1$\sigma$} & \textbf{Median} & \textbf{1$\sigma$} & \textbf{2$\sigma$} \\
       \hline
     \textbf{Expected limit} &0.011090 &  0.014832 & 0.020377 & 0.027946 & 0.037009 \\
     \hline
    \end{tabular}
    \caption{Expected upper limit on $BR(H\rightarrow \gamma \gamma_d)$ at 95\% CL.}
    \label{tab:limit_syst}
\end{table}

\vspace{2 cm}
\section{\textcolor{red}{To do}}
At the moment, only experimental systematics and data-driven background ones are included. Systematics from data-driven estimates are expected to be the dominant contribution (as already shown in the partial ranking plot).\\
For theory uncertainty, only needed for Zgamma, Wgamma and signals. Will use on-the-fly weights, already present in Ntuples. Workflow in place and preliminary estimates available. Will likely be included in the fit before the end of july.
A pruning threshold of 1\% is already applied. \\
The correlation scheme is also mostly defined: all uncertainties are correlated among different regions. Experimental systematics of MC are treated as correlated also among samples, while the systematics from data-driven backgrounds and the theory uncertainties will be uncorrelated among samples. Any non negligible correlation will be investigated in more details in the next few weeks.\\
At the moment, no smoothing is applied. The NPs behavior seems relatively stable, suggesting that there is no outstanding issue, but the distributions of the systematics will be checked and the smoothing will be applied using the TRexFitter utilities if any large statistical fluctuation is observed.\\
We aim at reaching a good understanding of the nuisance parameter behaviour (and more final treatment) by mid july. Overall, the fit looks healthy at this stage.\\
Work ongoing to include also Validation Regions in TRex-Fitter. Mostly done, but needs some cross-checks. \\
