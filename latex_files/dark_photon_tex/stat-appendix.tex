\subsection{Blind fit with pure Monte Carlo}
\label{sec:pureMC}

A first attempt of blind simultaneous fit has been done relying on Monte Carlo samples for the signal and for all the background processes. MC samples included in the fit are: 
\begin{itemize}
    \item for the signal: ggF, VBF, ZH, WH; 
    \item for the background: $W\gamma, Z\gamma,W$jets, $Z$jets, $\gamma$jets direct, $\gamma$jets fragmentation, dijets
\end{itemize}
A flat systematic of 10\% is applied to all the samples. 
The composition of SR and CRs in terms of samples is showed in Figure \ref{fig:pie_onlyMC}. 
\begin{figure}[!htbp]
    \centering
    \includegraphics[width=0.6\textwidth]{images/MConly/PieChart.pdf} 
    \caption{SR and CRs composition from Monte Carlo.}
    \label{fig:pie_onlyMC}
\end{figure}
\begin{figure}[!htbp]
    \centering

    \begin{minipage}[b]{0.32\textwidth}
        \includegraphics[width=\textwidth]{images/MConly/SR.pdf}
        \caption*{(a) Pre-fit SR}
    \end{minipage}
    \hfill
    \begin{minipage}[b]{0.32\textwidth}
        \includegraphics[width=\textwidth]{images/MConly/OneMuCR.pdf}
        \caption*{(b) Pre-fit $1\mu$CR}
    \end{minipage}
    \hfill
    \begin{minipage}[b]{0.32\textwidth}
        \includegraphics[width=\textwidth]{images/MConly/TwoMuCR.pdf}
        \caption*{(c) Pre-fit $2\mu$CR}
    \end{minipage}

    \vspace{0.5cm}

    \begin{minipage}[b]{0.32\textwidth}
        \includegraphics[width=\textwidth]{images/MConly/SR_postFit.pdf}
        \caption*{(d) Post-fit SR}
    \end{minipage}
    \hfill
    \begin{minipage}[b]{0.32\textwidth}
        \includegraphics[width=\textwidth]{images/MConly/OneMuCR_postFit.pdf}
        \caption*{(e) Post-fit $1\mu$CR}
    \end{minipage}
    \hfill
    \begin{minipage}[b]{0.32\textwidth}
        \includegraphics[width=\textwidth]{images/MConly/TwoMuCR_postFit.pdf}
        \caption*{(f) Post-fit $2\mu$CR}
    \end{minipage}

    \caption{Pre-fit and post-fit distributions using Monte Carlo only in SR and control regions.}
    \label{fig:prepost_onlyMC}
\end{figure}

Pre-fit and post-fit $m_T$ distribution in SR, $1\mu$CR and $2\mu$CR are showed in Figure \ref{fig:prepost_onlyMC}. Taking in consideration $m_T$ distribution in SR: 
\begin{itemize}
    \item $W$jets events are mainly located in the first bin, as these events are a background for the analysis only when the W decays in the electronic channel and the electron is mis-reconstructed as a photon; in such a case, the transverse mass $\gamma-p_T^{miss}$ is peaked on the W-boson mass ($\sim\SI{80}{GeV}$); 
    \item $Z\gamma$ events have a $m_T$ distribution quite similar to the one of the signal, being the irreducible background, with a peak in the central bin; 
    \item $W\gamma$ events contain real $p_T^{miss}$ so their spectrum reaches high values of $m_T$; 
    \item $\gamma$jets direct events are absent in the last bin as for such events the $p_T^{miss}$ is fake and is unlikely that a huge fake $p_T^{miss}$ is mis-reconstructed.
\end{itemize}
Regarding ratio plots: 
\begin{itemize}
    \item in SR, the ratio plots are empty as data are blind; 
    \item in $1\mu$CR, the ratio plot pre-fit presents the discrepancies between data and Monte Carlo, while post-fit Monte Carlo simulations are corrected with k-factors, so the ratio plot becomes flat; 
    \item in $2\mu$CR, the post-fit ratio plot is not flat as an inclusive $k_Z$ is used due to low statistics. 
\end{itemize}


k-factors are showed in Table \ref{tab:kfac_onlyMC}. $k_Z$ is particularly far from 1, quite unexpected for an electroweak background such as $Z\gamma$. This is mostly due to the under-estimation of jets faking photons contribution to $2\mu$CR by Monte Carlo simulations. 

\begin{table}[!htbp]
    \centering
    \begin{tabular}{c|c|c|c}
        \toprule
        $m_T$ \textbf{bin [GeV]} & \textbf{$k_W$} & \textbf{Error up} & \textbf{Error down}  \\
        \hline
         80-110 &  1.09996 &+0.139674 &-0.116532\\
      
        110-140 &  1.10981 &+0.103179 &-0.0936811\\
    
        140-200 & 1.22607 &+0.115021 &-0.0956884\\
    
        200-300 &  1.38527 &+0.15409 &-0.130573\\
  \hline
    $m_T$ \textbf{bin [GeV]} & \textbf{$k_Z$} & \textbf{Error up} & \textbf{Error down}  \\
    \hline
    inclusive & 1.41432& +0.155818 &-0.140022\\ 
    \bottomrule
    \end{tabular}
    \caption{k-factors in the only Monte Carlo configuration. }
    \label{tab:kfac_onlyMC}
\end{table}

The expected limit in this configuration is 0.015053. 
\begin{table}[!htbp]
    \centering
    \begin{tabular}{l
                    S[table-format=1.6]
                    S[table-format=1.6]
                    S[table-format=1.6]
                    S[table-format=1.6]
                    S[table-format=1.6]}
        \toprule
        & {Nominal} & {$-1\sigma$} & {$+1\sigma$} & {$-2\sigma$} & {$+2\sigma$} \\
        \midrule
        Expected limit &
        0.015053 & 0.010567 & 0.021724 & 0.007740 & 0.030515 \\
        
        \bottomrule
    \end{tabular}
    \caption{Expected limit in the only Monte Carlo configuration. }
\end{table}

\subsection{Blind fit including data-driven jet faking photon}
\label{sec:fit_fjy}
Jets faking photons background is then estimated with the data-driven technique explained in Section \ref{sec:jetfake} and included in the fit with its systematic. The 10\% flat systematic is then applied to all the samples, with the exception of jets faking photons. $\gamma$jets fragmentation, dijets and $W$jets, $Z$jets without electrons faking photons have been then removed from the fit, while including the data-driven estimation. $W$jets, $Z$jets with electrons faking photons are selected requiring the truth type of the photon being equal to 2 (i.e. electron). This leads to a varied composition of backgrounds in SR, where jets faking photons account for a larger portion of background, as it can be seen in Figure \ref{fig:pie_jfy}. 
\begin{figure}[!htbp]
    \centering
    \includegraphics[width=0.6\textwidth]{images/jfy/PieChart.pdf} 
    \caption{SR and CRs composition.}
    \label{fig:pie_jfy}
\end{figure}
Pre-fit and post-fit $m_T$ distribution in SR, $1\mu$CR and $2\mu$CR are showed in Figure \ref{fig:prepost_jfy}. Adding data-driven jets faking photons, $Z$jets contribution to background is very low, so that is not visible in the plots; this is expected as this events enter the SR only when multiple things happen at the same time: the Z boson decays in the electronic channel, an electron is lost giving a fake $p_T^{miss}$, the other electron is mis-reconstructed as photon. 
\begin{figure}[!htbp]
    \centering

    % Prima riga
    \begin{minipage}[b]{0.32\textwidth}
        \includegraphics[width=\textwidth]{images/jfy/SR.pdf}
        \caption*{(a) Pre-fit SR}
    \end{minipage}
    \hfill
    \begin{minipage}[b]{0.32\textwidth}
        \includegraphics[width=\textwidth]{images/jfy/OneMuCR.pdf}
        \caption*{(b) Pre-fit $1\mu$CR}
    \end{minipage}
    \hfill
    \begin{minipage}[b]{0.32\textwidth}
        \includegraphics[width=\textwidth]{images/jfy/TwoMuCR.pdf}
        \caption*{(c) Pre-fit $2\mu$CR}
    \end{minipage}

    \vspace{0.4cm}

    % Seconda riga
    \begin{minipage}[b]{0.32\textwidth}
        \includegraphics[width=\textwidth]{images/jfy/SR_postFit.pdf}
        \caption*{(d) Post-fit SR}
    \end{minipage}
    \hfill
    \begin{minipage}[b]{0.32\textwidth}
        \includegraphics[width=\textwidth]{images/jfy/OneMuCR_postFit.pdf}
        \caption*{(e) Post-fit $1\mu$CR}
    \end{minipage}
    \hfill
    \begin{minipage}[b]{0.32\textwidth}
        \includegraphics[width=\textwidth]{images/jfy/TwoMuCR_postFit.pdf}
        \caption*{(f) Post-fit $2\mu$CR}
    \end{minipage}

    \caption{Pre-fit and post-fit distributions in SR and CRs using data-driven $j \to \gamma$ estimation.}
    \label{fig:prepost_jfy}
\end{figure}

k-factors are showed in Table \ref{tab:kfac_jfy}. $k_Z$ is now closer to 1 with reference to the only Monte Carlo configuration, thanks to the data-driven estimation of jets faking photons. 
\begin{table}[!htbp]
    \centering
    \begin{tabular}{c|c|c|c}
        \toprule
        $m_T$ \textbf{bin [GeV]} & \textbf{$k_W$} & \textbf{Error up} & \textbf{Error down}  \\
        \hline
         80-110 & 0.826057 &+0.119916 &-0.103399\\
      
        110-140 & 0.831074 & +0.0905051 & -0.0871422 \\
    
        140-200 & 0.89956 &  +0.0983367 & -0.0859015\\
    
        200-300 &  1.1688 & +0.144942 & -0.124162\\
  \hline
    $m_T$ \textbf{bin [GeV]} & \textbf{$k_Z$} & \textbf{Error up} & \textbf{Error down}  \\
    \hline
    inclusive & 0.8081 &+0.131243 &-0.143476\\ 
    \bottomrule
    \end{tabular}
        \caption{k-factors including data-driven jets faking photons. }
    \label{tab:kfac_jfy}
\end{table}
The expected limit in this configuration is 0.016014. 
\begin{table}[!htbp]
    \centering
    
    \begin{tabular}{l
                    S[table-format=1.6]
                    S[table-format=1.6]
                    S[table-format=1.6]
                    S[table-format=1.6]
                    S[table-format=1.6]}
        \toprule
        & {Nominal} & {$-1\sigma$} & {$+1\sigma$} & {$-2\sigma$} & {$+2\sigma$} \\
        \midrule
        Expected limit &
        0.016014 & 0.011235 & 0.023131 & 0.008226 & 0.032520 \\
        
        \bottomrule
    \end{tabular}
    \caption{Expected limit including jets faking photons data-driven estimation.}
\end{table}

\subsection{Blind fit with data-driven backgrounds and BDT score cut}

This section displays the results of a blind simultaneous fit in which data-driven estimations of $j \rightsquigarrow \gamma$ and $e \rightsquigarrow \gamma$ backgrounds (see \ref{sec:jetfake} and \ref{sec:elefake} for details) are included, together with the selection applied on the BDT score obtained in section \ref{sec:vertex}. The fit takes into consideration the systematic uncertainties on the data-driven $j \rightsquigarrow \gamma$ and $e \rightsquigarrow \gamma$ estimations, and a "guess" 10\% flat systematic, applied to all other backgrounds so as to mimic the foreseen effects of the experimental and theoretical uncertainties that will be included later.\\
To allow for the calculation of the exclusion limit, the configuration is a \textit{signal+background} (SPLUSB) fit, in which the POI was set to 0 in order to put ourselves in the case with no injected signal.\\
Comparing the background composition in the three regions resulting from a fit with MC-estimated $e \rightsquigarrow \gamma$ (\ref{sec:stat-appendix}) and data-driven $e \rightsquigarrow \gamma$ (figure \ref{fig:piecharts}), it emerges that a data-driven estimation leads to $e \rightsquigarrow \gamma$ being the dominant background in SR, much differently from the case in which it was taken from MC simulations.\\

\begin{figure}[h!]
\centering
    \begin{minipage}{0.45\textwidth}
    \centering
    \includegraphics[width=\linewidth]{images/PieChart.pdf}
    \caption*{(a) Pre-fit pie charts.}
    \label{fig:pie_pre}
    \end{minipage}
\quad
    \begin{minipage}{0.45\textwidth}
    \centering
    \includegraphics[width=\linewidth]{images/PieChart_postFit.pdf}
    \caption*{(b) Post-fit pie charts.}
    \label{fig:pie_post}
    \end{minipage}

\caption{Pre-fit and post-fit SR and CRs composition.}
\label{fig:piecharts}

\end{figure}

\noindent
The 1$\mu$CR is populated by $e\rightsquigarrow\gamma$, differently from the pure MC case (section ??) due to processes such as $W(\rightarrow e\nu)W(\rightarrow \mu\nu)$ which are not used among the MC sample. 

\vspace{1 cm}

\noindent
Numerically, the background yields displayed in the pie charts are summarized in tables \ref{tab:yields_pre} and \ref{tab:yields_post}.

\begin{table}[h!]
    \centering
    \begin{tabular}{|c|c|c|c|}
    \hline
       \textbf{Process}  & \textbf{SR} & \textbf{1$\mu$CR} & \textbf{2$\mu$CR} \\
    \hline
    $ggF_{dark}$ & 0 $\pm$ 0& 0 $\pm$ 0 & 0 $\pm$ 0\\
    \hline
     $VBF_{dark}$  & 0 $\pm$ 0 & 0 $\pm$ 0 & 0 $\pm$ 0 \\
    \hline
      $WH_{dark}$ & 0 $\pm$ 0 & 0 $\pm$ 0 & 0 $\pm$ 0 \\
    \hline
      $ZH_{dark}$ & 0 $\pm$ 0 & 0 $\pm$ 0 & 0 $\pm$ 0 \\
    \hline
      $W\gamma$ & 450.797 $\pm$ 45.6681& 739.316 $\pm$ 74.3611& 0.203071 $\pm$ 0.0207582\\
    \hline
      $Z\gamma$ & 251.242 $\pm$ 25.5152& 21.7076 $\pm$ 2.1859& 87.9189 $\pm$ 8.91118\\
    \hline
      $\gamma$jets(direct) &48.0859 $\pm$ 5.04346 &0.0358423 $\pm$ 0.00365839 & 3.94582e-06 $\pm$ 6.1096e-08\\
    \hline
     $j \rightsquigarrow \gamma$  &619.18 $\pm$ 105.79 & 263.35 $\pm$ 45.34& 53.22 $\pm$ 9.46\\
    \hline
     $e \rightsquigarrow \gamma$  & 835.789 $\pm$ 81.0756& 68.2427 $\pm$ 6.75506& 4e-06 $\pm$ 0\\
    \hline
     Total  &2205.09 $\pm$ 153.494 & 1092.65 $\pm$ 89.2254& 141.342 $\pm$ 13.0103\\
    \hline
     Data  & 0 & 937& 124\\
    \hline
    
    \end{tabular}
    \caption{Pre-fit yields.}
    \label{tab:yields_pre}
\end{table}


\begin{table}[h!]
    \centering
    \begin{tabular}{|c|c|c|c|}
    \hline
       \textbf{Process}  & \textbf{SR} & \textbf{1$\mu$CR} & \textbf{2$\mu$CR} \\
    \hline
    $ggF_{dark}$ & 0.154324 $\pm$ 19.3776& 0 $\pm$ 0 & 0 $\pm$ 0\\
    \hline
     $VBF_{dark}$  & 0.0372225 $\pm$ 4.67381 & 0 $\pm$ 0 & 0 $\pm$ 0 \\
    \hline
      $WH_{dark}$ & 0.00935622 $\pm$ 1.1748 & 0.000513275 $\pm$ 0.0644495  & 0 $\pm$ 0 \\
    \hline
      $ZH_{dark}$ & 0.00610652 $\pm$ 0.76676 &  2.56389e-05 $\pm$ 0.00321934 & 0 $\pm$ 0 \\
    \hline
      $W\gamma$ & 345.859 $\pm$ 31.2352& 581.178 $\pm$ 48.6794& 0.154028 $\pm$ 0.014861\\
    \hline
      $Z\gamma$ & 194.355 $\pm$ 37.8637 & 16.7927 $\pm$ 3.27518 & 68.0693 $\pm$ 13.2259\\
    \hline
      $\gamma$jets(direct) & 48.0868 $\pm$ 4.92906 & 0.0358432 $\pm$ 0.00363269 & 3.9456e-06 $\pm$ 6.10261e-08\\
    \hline
     $j \rightsquigarrow \gamma$  & 636.495 $\pm$ 94.3173 & 270.771 $\pm$ 40.4266 & 54.7681 $\pm$ 8.43809\\
    \hline
     $e \rightsquigarrow \gamma$  & 835.7 $\pm$ 45.3879 & 68.2354 $\pm$ 3.78041 & 4e-06 $\pm$ 0 \\
    \hline
     Total  &2060.7 $\pm$ 45.0162 & 937.013 $\pm$ 29.8062 & 122.991 $\pm$ 9.63825\\
    \hline
     Data  & 0 & 937& 124\\
    \hline
    
    \end{tabular}
    \caption{Post-fit yields.}
    \label{tab:yields_post}
\end{table}

\noindent
The plots in figure \ref{fig:regions_jy_ey_Fit} show the pre- and post-fit $m_T$ distributions in SR, 1$\mu$CR and 2$\mu$CR:\\

\begin{figure}[h!]
\centering
    \begin{minipage}{0.3\textwidth}
    \centering
    \includegraphics[width=\linewidth]{images/SR.pdf}
    \caption*{(a) Pre-fit SR.}
    \label{fig:SR}
    \end{minipage}
\quad
    \begin{minipage}{0.3\textwidth}
    \centering
    \includegraphics[width=\linewidth]{images/OneMuCR.pdf}
    \caption*{(b) Pre-fit 1$\mu$CR.}
    \label{fig:1mu}
    \end{minipage}
\quad
    \begin{minipage}{0.3\textwidth}
    \centering
    \includegraphics[width=\linewidth]{images/TwoMuCR.pdf}
    \caption*{(c) Pre-fit 2$\mu$CR.}
    \label{fig:2mu}
    \end{minipage}


    \begin{minipage}{0.3\textwidth}
    \centering
    \includegraphics[width=\linewidth]{images/SR_postFit.pdf}
    \caption*{(d) Post-fit SR.}
    \label{fig:SR_post}
    \end{minipage}
\quad
    \begin{minipage}{0.3\textwidth}
    \centering
    \includegraphics[width=\linewidth]{images/OneMuCR_postFit.pdf}
    \caption*{(e) Post-fit 1$\mu$CR.}
    \label{fig:1mu_post}
    \end{minipage}
\quad
    \begin{minipage}{0.3\textwidth}
    \centering
    \includegraphics[width=\linewidth]{images/TwoMuCR_postFit.pdf}
    \caption*{(f) Post-fit 2$\mu$CR.}
    \label{fig:2mu_post}
    \end{minipage}

\caption{Pre-fit and post-fit distributions in SR and CRs.}
\label{fig:regions_jy_ey_Fit}
    
\end{figure}

\vspace{5cm}
\noindent
using a binned $k_W$ allows to have a data/background ratio exactly equal to 1, while an inclusive $k_Z$ provides a less precise ratio, but still within 1$\sigma$. The values of the k-factors and their related statistical uncertainties are listed in table \ref{tab:kfactors}.

\begin{table}[h!]
    \centering
    \begin{tabular}{|c|c|c|c|}
    \hline
      \textbf{$m_T$ bin [GeV]} & \textbf{$k_W$} & \textbf{Error up} & \textbf{Error down} \\
    \hline
      80-110 & 0.710612 & +0.135519 & -0.11107  \\
    \hline
      110-140 & 0.69346 & +0.118731 &  -0.10161 \\
    \hline
      140-200   & 0.802358 & +0.135485 & -0.109737 \\
    \hline
     200-300 & 1.10656 & +0.174307 & -0.145853 \\
    \hline
     \textbf{$m_T$ bin [GeV]} & \textbf{$k_Z$} & \textbf{Error up} & \textbf{Error down}\\
    \hline
     Inclusive & 0.773567 & +0.183894 & -0.158675 \\
    \hline
    \end{tabular}
    \caption{Normalization factors.}
    \label{tab:kfactors}
\end{table}

\noindent
The expected upper limit on $BR(H\rightarrow \gamma \gamma_d)$ in this configuration is 0.019642 (see table \ref{tab:limit}):\\

\begin{table}[h!]
    \centering
    \begin{tabular}{|c|c|c|c|c|c|}
    \hline
       & \textbf{-2$\sigma$} & \textbf{-1$\sigma$} & \textbf{Median} & \textbf{1$\sigma$} & \textbf{2$\sigma$} \\
       \hline
     \textbf{Expected limit} & 0.010329 & 0.013982 & 0.019642 & 0.027846 & 0.038455 \\
     \hline
    \end{tabular}
    \caption{Expected upper limit on $BR(H\rightarrow \gamma \gamma_d)$ at 95\% CL.}
    \label{tab:limit}
\end{table}

\vspace{2 cm}
\noindent
The value of the limit is slightly lowered by the BDT selection applied in the three regions, with respect to the case in which it wasn't used.\\
\noindent
Figures \ref{fig:pullplot} and \ref{fig:corr} display the pull plot and the correlation matrix, respectively, referred to the nuisance parameters involved in the fit.

\begin{figure}[h!]
    \centering
    \includegraphics[width=0.7\linewidth]{images/NuisPar.pdf}
    \caption{Nuisance parameters pull plot: \texttt{ff\_jy} and \texttt{ff\_ey} are the systematic uncertainties on the data-driven $j \rightsquigarrow \gamma$ and $e \rightsquigarrow \gamma$ fake rates, while \texttt{fey\_syst} is the systematic on the trigger scale factor for $e \rightsquigarrow \gamma$ estimation; \texttt{guess\_syst} refers to the 10\% guess systematic on MC backgrounds.\\
    $[$Comment: TRExFitter seems not to include the inclusive normalization factor kz in the pull plot, while it is displayed in the correlation matrix$]$.}
    \label{fig:pullplot}
\end{figure}

\begin{figure}[h!]
    \centering
    \includegraphics[width=0.7\linewidth]{images/CorrMatrix.pdf}
    \caption{Correlation matrix: \texttt{ff\_jy} and \texttt{ff\_ey} are the systematic uncertainties on the data-driven $j \rightsquigarrow \gamma$ and $e \rightsquigarrow \gamma$ fake rates, while \texttt{fey\_syst} is the systematic on the trigger scale factor for $e \rightsquigarrow \gamma$ estimation; \texttt{guess\_syst} refers to the 10\% guess systematic on MC backgrounds.}
    \label{fig:corr}
\end{figure}





