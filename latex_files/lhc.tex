The LHC (insert reference here), located near Geneva, Switzerland, is the world's largest and most powerful hadron accelerator. 
It collides protons at a center-of-mass (COM) energy of 13.6 TeV during Run 3 operations, with plans to reach its design energy of 14 TeV in the High-Luminosity LHC (HL-LHC) era.
The two opposing beams of protons travel in a 27-kilometer ring of super-conducting magnets and are made to collide at four evenly spaced interaction points highlighted in Fig. \ref{accelerator_complex} every 25 ns.
There are 2 general-purpose detectors, ATLAS \cite{PERF-2007-01} and CMS \cite{CMS-CMS-00-001}, and 2 specialized detectors, LHCb (ref) and ALICE(ref), located at the interaction points to observe and record the results of the proton-proton collisions.

Currently in Run 3, the LHC has operated since 2009, with two long shutdowns between runs to allow for upgrades to the accelerator complex and detectors.
Run 1 took place from 2010 to 2013 at center-of-mass energies of 7 and 8 TeV, while Run 2 took place from 2015 to 2018 at a center-of-mass energy of 13 TeV.
The High-Luminosity LHC (HL-LHC) is scheduled to begin operations in  2030 after a third long shutdown to upgrade the LHC and detectors to handle and take advantage of the large increase in luminosity.

\begin{figure}[htbp]
    \centering
    \includegraphics[width=0.8\textwidth]{../images/my_images/accelerator_complex.pdf}
    \caption{CERN accelerator complex\cite{VandenBroeck:2019CERNComplex}. The four main experiments are located at the interaction points highlighted in yellow. As of Run 3, the proton acceleration chain begins at LINAC4 highlighted in purple.}
    \label{accelerator_complex}
\end{figure}

\subsection{The Proton Acceleration Chain}\label{subsec:proton-chain}
The CERN accelerator complex prepares and accelerates protons before injecting them into the LHC as seen in Fig. \ref{accelerator_complex}.
Prior to Run 3, hydrogen gas is ionized to produce protons, which are then accelerated to an energy of 50 MeV in the linear accelerator (LINAC2) before being injected into the Proton Synchrotron Booster (PSB).
Run 3 replaced LINAC2 with LINAC4, which accelerates negative hydrogen ions (H-) to 160 MeV before being injected into the PSB.
Next, a series of synchrotrons further the acceleration process: the PSB accelerates protons to 2 GeV, followed by the Proton Synchrotron (PS) with 26 GeV, and finally the Super Proton Synchrotron (SPS) where protons reach an energy of 450 GeV.
Within the PS, the protons are grouped into bunches containing ~$10^{11}$ protons each, and these bunches are split into opposing beams in the SPS.
The two beams are finally ready to be injected into the LHC for the final acceleration up to 6.5 TeV per beam in Run 2, and 6.8 TeV per beam in Run 3.
The beams collide at the 4 interaction points every 25 ns, resulting in a maximum of 40 million collisions per second. 

Each system in the chain uses radio-frequency (RF) cavities for acceleration, while the synchrotrons also include magnetic dipoles for beam steering and magnetic quadrupoles for focusing.
Since protons are charged particles, RF cavities use oscillating electric fields to accelerate them.
These cavities are shaped to resonate at frequencies that match the timing of the proton bunches, accelerating them forward while keeping the bunches tightly packed. 
The frequency is gradually increased as the protons gain energy, imparting a "pull" as they enter the cavity and a "push" as they exit.
The bunches are shaped longitudinally by the RF cavities.
This is done by having the protons encounter the fields at the crest of the wave, slightly pushing the forward protons back and the trailing protons forward, keeping the bunches compact.
Opposite polarity fields are applied by dipole magnets shown in Fig. (insert refrence and figure) bend the beam trajectories along the ring in opposing directions.








