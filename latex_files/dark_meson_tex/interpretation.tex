The results are used to set $95\%$ confidence level (CL) upper limits on the dark pion pair production cross-sections following the $\text{CL}_S$ technique~\cite{Read:2002hq}. 
Upper limits on the dark pion production cross-sections are shown in Figure~\ref{fig:allhad_limits_unblinded_su2l_1d} for four slices of \etaD.
The impacts of the different systematic uncertainties, described in Section~\ref{sec:systematics}, on the signal strength, defined as the signal cross-section normalized to the theoretical prediction, of two benchmark signals are summarized in Table~\ref{tab:syst_table}: the analysis is limited by the systematic uncertainty, with dominant contribution from the theoretical sources of uncertainty on the background modeling.

\begin{table}[htbp]
  \caption{Impact of different categories of systematic uncertainty in the \onelep channel, for two signal benchmarks, relative to the total uncertainty on the fitted signal strength. For each category, the fit is repeated with the corresponding group of nuisance parameters fixed to their best-fit values and the impact for each category is evaluated as the quadrature difference between the signal strength uncertainty in the new fit and in the nominal one, divided by the uncertainty in the nominal fit. The contribution from the statistical uncertainty and the systematic one, further separated into the global instrumental and theoretical uncertainties, are shown. The total systematic uncertainty is different from the sum in quadrature of the different groups due to the correlations among the nuisance parameters in the fit.
  }
  \label{tab:syst_table}
  \centering
  \begin{tabular}{l|cc}
    & \sul, $\etaD=0.25$, & \sul, $\etaD=0.35$, \\
   Category & $m_{\pi_{D}}=400~\GeV$ & $m_{\pi_{D}}=700~\GeV$ \\
  \toprule
  Luminosity  &  0.03  &0.05\\
Pileup  &  0.05 &0.09 \\
Flavor tagging & 0.28 & 0.26 \\
Leptons & 0.01 & 0.04 \\ 
Jets  &  0.08  &0.14 \\
\ttb &   0.26  &0.53\\
\ttc  &  0.12  & 0.18\\
\ttlight  &  0.13 & 0.17 \\
Top \pt NNLO reweighting  &  0.08  &0.09 \\
Single top  &  0.06  &0.06\\
\hline
Statistical &   0.28 & 0.24 \\
% Systematics & 0.61 & 0.81 \\
Instrumental & 0.30& 0.30\\
Theory & 0.38 & 0.63\\
  %Statistical &   2.3 & 13 \\
  %Instrumental & 2.5 & 16\\
  %Theory & 2.9 & 35\\
  \bottomrule
  \end{tabular}
  \end{table}

Using the predicted dark pion pair production cross-sections, the limits can be translated into limits on dark pion masses in the two-dimensional $\etaD\text{--}\dpmass$ plane. The exclusion contour obtained from the \allhad channel for the \sul model is shown in Figure~\ref{fig:allhad_limits}. 
%Figure~\ref{fig:allhad_limits_unblinded_su2l_1d} shows the \sul exclusion limits for dark pions at four different $\eta$ values. 
No dark pion masses can be excluded for $\etaD=0.15$. 
For $\etaD=0.25$ the exclusion covers the mass range $280~\GeV<\dpmass<520~\GeV$ (expected $280~\GeV<\dpmass<540~\GeV$), while for $\etaD=0.35$, dark pions with masses $\dpmass<430~\GeV$ are excluded (expected $\dpmass<450~\GeV$). %For $\eta=0.45$ the excluded range cannot be extended beyond the range already excluded by reinterpretation of other ATLAS or CMS analyses~\cite{Kribs:2018ilo}. 
%The excluded region can also be plotted in the two-dimensional $\eta\text{--}\dpmass$--plane, which was done in figure~\ref{fig:allhad_unblinded_limits_su2l_2d} and shows again a good agreement between the expected and the observed exclusion limits.
%  Similar to the \allhad channel, the results are used to set $95\%$ CL upper limits on the dark pion production cross-sections as well as limits on dark pion masses in the two-dimensional $\eta\text{--}\dpmass$ plane. 
The exclusion contour obtained from the \onelep channel for \sul is shown in Figure~\ref{fig:1lep_limits} and is observed to fully cover the \allhad limit and significantly extend the probed phase space for this dark meson model. Since the \allhad limit is completely contained within the \onelep limit, there is no expected gain from a combination of the two channels.
% Upper limits on
% the production cross-section as a function of the dark pion  mass are therefore derived and are shown in Figure~\ref{fig:1lep_xsec_limit}.
% l and \sur model respectively, in four slices of $\eta$ corresponding to the values $\eta=0.45$, $0.35$, $0.25$ and $0.15$. 
Dark pion masses with $\dpmass<940~\GeV$ can be excluded for $\etaD=0.45$, $\dpmass<720~\GeV$ are excluded for $\etaD=0.35$ and for $\etaD=0.25$ the mass region $\dpmass<740~\GeV$ is excluded. The results significantly extend the phase space previously excluded through re-interpretation of other collider searches~\cite{Butterworth:2021jto}.
% For the \sur models, the upper limits obtained from both channels are observed to be systematically above the theory expectations, thus showing that also the \onelep channel is not sensitive to this model with the currently available dataset. 
In the \sur model, cross sections are significantly smaller than for \sul and therefore none of the channels have sensitivity to this model with the current data sample.

\begin{figure}[btp]
  \centering
  \subfloat[]{
    \centering
    \includegraphics[width=0.445\textwidth]{limit_1lep_SU2L_gaugephobic_45.pdf}\label{figaux:eta45}
  }
  \subfloat[]{
    \centering
    \includegraphics[width=0.445\textwidth]{limit_1lep_SU2L_gaugephobic_35.pdf}\label{figaux:eta35}
  } \\
  \subfloat[]{
    \centering
    \includegraphics[width=0.445\textwidth]{limit_1lep_SU2L_gaugephobic_25.pdf}\label{figaux:eta25}
  }
  \subfloat[]{
    \centering
    \includegraphics[width=0.445\textwidth]{limit_1lep_SU2L_gaugephobic_15.pdf}\label{figaux:eta15}
  }
  \caption{Observed (solid line) and expected (dashed line) limits on the dark pion production cross-section as a function of dark pion mass using the $\text{CL}_S$ method for all \sul models in four slices of $\etaD$: \protect\subref{figaux:eta45} $\etaD$=0.45, \protect\subref{figaux:eta35} $\etaD$=0.35, \protect\subref{figaux:eta25} $\etaD$=0.25, and \protect\subref{figaux:eta15} $\etaD$=0.15. The surrounding shaded bands correspond to one and two standard deviations around the expected limit. The overlaid theory line shows the theoretical dark pion cross-section prediction~\cite{Kribs:2018ilo}.}
  \label{fig:allhad_limits_unblinded_su2l_1d}
\end{figure}

\begin{figure}[btp]
  \centering
  \subfloat[]{\includegraphics[width=0.495\textwidth]{limit_allhad_unblinded_2d_SU2L_gaugephobic.pdf}\label{fig:allhad_limits}}
  \subfloat[]{\includegraphics[width=0.495\textwidth]{limit_2d_1lep_SU2L_gaugephobic.pdf}\label{fig:1lep_limits}}
  \caption{Observed (solid line) and expected (dashed line) exclusion contours at $95\%$ CL in the $\etaD\text{--}m_{\pi_D}$ plane for \sul signal models in the \protect\subref{fig:allhad_limits} \allhad and \protect\subref{fig:1lep_limits} \onelep channel. Masses that are within the contours are excluded, as indicated by the hatched area. An uncertainty band corresponding to the $\pm 1\sigma$ variation on the expected limit is also indicated. The shaded area in \protect\subref{fig:allhad_limits} and the innermost shaded area in \protect\subref{fig:1lep_limits} indicates the phase space previously excluded through re-interpretation of other collider searches presented in Ref.~\cite{Butterworth:2021jto}. The outermost shaded area in \protect\subref{fig:1lep_limits} indicates the phase space excluded by the analysis in the \allhad channel and is identical to the observed limit shown in \protect\subref{fig:allhad_limits}.}
  \label{fig:
  limits}
\end{figure}
\FloatBarrier
