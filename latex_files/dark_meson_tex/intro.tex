%!TEX root = ../ANA-EXOT-2023-09-PAPER.tex

The Standard Model (SM) can be extended by a new strongly-coupled, confining gauge theory with fermion representation which transforms under the electroweak group.
The appeal of such an extension is that dark matter can arise in the form of composite mesons or baryons of the new strongly-coupled theory. In addition, these models often exhibit an automatic accidental symmetry protecting against dark matter decay.
%Consequently, a vast amount of research has been put into the study of strongly-coupled dark matter manifesting itself as dark mesons, dark baryons or dark quarkonia-like states~\cite{Kribs:2016cew}; one set of models incorporating this concept is Stealth Dark Matter~\cite{Appelquist:2015yfa}. 
Consequently, candidates for strongly-coupled dark matter include dark mesons, dark quarkonia-like states, glueballs and dark baryons~\cite{Kribs:2016cew,Butterworth:2021jto}.
The search presented here targets one set of models incorporating this concept: Stealth Dark Matter~\cite{Appelquist:2015yfa}.
Here, the new strongly-coupled dark sector consists of vector-like fermions that can transform under the new dark group but also interact with both the electroweak sector of the SM and the Higgs boson.
The result is the emergence of a quantum chromodynamics (QCD)-like dark sector as the direct analog to the QCD meson and baryon sector.
This leads to several intriguing phenomenological consequences:
as long as the vector-like mass is dominant over the chiral mass, the new dark sector is only weakly constrained by precision electroweak or Higgs coupling measurements, while the Higgs interactions break the dark sector global symmetry and thus allow dark mesons to decay into pure SM states~\cite{Kribs:2018ilo}.
This search focuses on the low-energy effective theories developed in Ref.~\cite{Kribs:2018oad}, which incorporate the leading interactions between dark mesons of a strongly-coupled $SU(2)$ dark flavor symmetry preserving dark sector and the SM.
These models contain a stable dark scalar baryon which could account for the stable dark matter observed in cosmological measurements~\cite{Appelquist:2015yfa}.

The simplified model targeted in this search contains only the two phenomenologically relevant sets of dark mesons: a lighter pseudoscalar triplet of dark pions, \piD, and an additional triplet of dark rho vector mesons, \rhoD, which are both expected at a scale around or slightly above the electroweak scale.
Following standard theoretical assumptions, the triplets are completely mass-degenerate and the dark sector can be fully described by three parameters: the mass of the dark pions $m_{\pi_D}$, the mass of the dark rho mesons $m_{\rho_D}$, and the number of dark colors $N_D$.
Since the phenomenological consequences remain unchanged for values of $N_D$ that are not excessively large, $N_D$ is fixed to $N_D=4$ throughout this search following the typical choice made for Stealth Dark Matter~\cite{Appelquist:2015yfa}. 

%The phenomenological details (e.g., the Lagrangian and additional explanation) can be found in Ref.~\cite{Kribs:2018ilo}; a summary is provided here.

\begin{figure}[tbp]
  % The Feynman diagrams were produced in LaTeX and are defined in FeynmanDiagram.tex, which can be found in the dark meson git space: https://gitlab.cern.ch/darkmesonsearch/darkscripts/tree/master/FeynmanDiagrams
  \centering
  \subfloat[]{\includegraphics[page=6,width=0.31\textwidth]{../images/dark_meson/FeynmanDiagram.pdf}\phantomsubcaption\label{fig:FeynmanW}} \hspace{2mm}
  %\subfloat[]{\includegraphics[page=7,width=0.32\textwidth]{FeynmanDiagram.pdf}\label{fig:FeynmanB}} %\hspace{20mm}
  \subfloat[]{\includegraphics[page=2,width=0.31\textwidth]{../images/dark_meson/FeynmanDiagram.pdf}\phantomsubcaption\label{fig:FeynmanDY}} \hspace{2mm}
  \subfloat[]{\includegraphics[width=0.31\textwidth]{../images/dark_meson/pionttbb.pdf}\phantomsubcaption\label{fig:FeynmanDecay}} %\hspace{20mm}
  \caption{Examples of leading Feynman diagrams of dark pion pair production. The diagram in \protect\subref{fig:FeynmanW} shows resonant production via kinetic mixing with the \Wboson-field resulting in either a neutral or charged dark rho meson, a mixing with the $B$-field that can only result in a neutral dark rho meson is also possible, \protect\subref{fig:FeynmanDY} shows Drell--Yan-type pair production of dark pions, and \protect\subref{fig:FeynmanDecay} shows an illustrative diagram of the dark pion decay into a top and a bottom quark for dark pion production.}
  \label{fig:feynmangraphs}
\end{figure}

Contrary to QCD, the vector-like nature of the dark sector allows to either gauge the full \sul weak interaction symmetry group or just the underlying $U(1)$ group, which leads to two distinct models of kinetic mixing of dark mesons with the SM, \sul and \sur. 
% This search focuses on gaugephobic \sul models, where the preferred decay channels are to pairs of fermions and which result in considerably larger cross sections than \sur models.
%We are distinguishing the two models of kinetic mixing by the labels \sul and \sur respectively.
The phenomenological consequences manifest themselves in the allowed decay channels and production cross-sections of dark pions, where the \sul models result in considerably larger cross sections than the \sur models.
%Dark pions are always pair-produced either via Drell-Yan processes, where electroweak gauge bosons kinetically mix with the dark pions, or resonantly via the production of a dark rho which then subsequently decays into a pair of dark pions.
Dark pions are always pair-produced either via Drell--Yan-type processes or resonantly via kinetic mixing of SM electroweak gauge bosons with the \rhoD that then subsequently decays into a pair of dark pions, as shown in 
Figure~\ref{fig:feynmangraphs}. The kinetic mixing parameter $\epsilon$ 
%and $\epsilon^\prime$, corresponding to \sul and \sur, respectively, 
depends on the number of dark colors as shown in Figure~\ref{fig:FeynmanW} (see also Ref.~\cite{Kribs:2018ilo}).
Throughout nearly all of the parameter space investigated in this search, the resonant production dominates the production of dark pions.
Once the choice of $N_D=4$ is made, the production cross-section depends trivially on the ratio of the dark pion and dark rho-meson masses, for which the symbol $\etaD=m_{\piD}/m_{\rhoD}$ is used, equivalent to the $\eta$ defined in Ref.~\cite{Kribs:2018ilo}. For gaugephobic \sul models, a given dark pion mass and \etaD-parameter fully specify the model, including the dark pion decay branching fractions.
%\sul models can be further classified into two categories depending on whether the dark pion decays are gaugephobic, i.e. the preferred decay channels are to pairs of fermions, or gaugephilic, where the decay to Higgs, $W$ and $Z$ dominates if kinematically allowed.
%In \sur models dark pions always decay gaugephobically.
%Consequently, there are three fundamentally distinct models of dark mesons to consider for a search.

%In all cases dark pions decay promptly back into pure SM states.

This search considers only models with $\etaD<0.5$ where the decay $\rho_D^{\pm,0}\rightarrow\pi_D^{\pm}\pi_D^{0,\mp}$ has a branching fraction of nearly $1.0$, while for models with $\etaD>0.5$ this decay is kinematically forbidden and the dark rho meson can only decay to pairs of leptons or quarks.
Previous searches for resonances in the dilepton spectrum both in ATLAS~\cite{EXOT-2016-05} and CMS~\cite{CMS-EXO-16-047} have placed strong constraints on such models~\cite{Kribs:2018ilo}.
The bounds for models with $\etaD<0.5$ are considerably weaker~\cite{Butterworth:2021jto}.
This is the first search in any collider experiment optimized for this specific type of model. %Other searches fail to be sensitive to these models because they either require a large amount of missing transverse energy, consider single-production, or are optimized for high-mass resonances and do not retain sensitivity to the mass regime $300-600~\GeV$ relevant for the models investigated in this search~\cite{Kribs:2018ilo}.

Figure~\ref{fig:crossections} shows the pair-production cross-sections for dark pions in \sul and \sur models.
%In general the cross sections are larger for the \sul scenario, but get smaller the further the meson masses move away from the resonance at $\etaD=0.5$.
The contribution of resonant production to the total production cross-section is indicated by the dashed lines.
%The mass points at $\etaD=0.15$ in \sul and \sur models are the only samples for which the Drell-Yan production mode dominates over the resonant production.
%This has kinematic consequences as the dark pions are much softer compared to those produced in the decay of a very heavy intermediate particle. 
%For convenience, the cross sections for the signal points investigated in this analysis have been listed explicitly in appendix~\ref{app:cross_sections}.
A variety of different decay channels are open to dark pions in the available parameter space.
The most relevant channels and their branching fractions are shown in Figure~\ref{fig:branchingfractions}.
For gaugephobic models the decay to top and bottom quarks dominates at high masses, while decays to bottom and charm quarks, $\tau$-leptons and gauge bosons are relevant at lower dark pion masses. 
%For gaugephilic models the decay behaviour is more complex as the decays to gauge bosons and to top and bottom quarks simultaneously exhibit large branching fractions.
%This leads to very complex final states with several bosons, $b$-jets and top decays in the same event.
%For reasons of simplicity we will therefore focus exclusively on \sul gaugephobic models in the remainder of this search and specifically target final states containing either three top- and one bottom-quark or two top- and two bottom-quarks.

\begin{figure}[tbp]
  % The plots have been produced with the plot_crossSections.C script which can be found in the dark meson git space: https://gitlab.cern.ch/darkmesonsearch/darkscripts/blob/master/plot_crossSections.C
  \centering
  %\includegraphics[width=0.32\textwidth]{crossSections_darkRho.pdf}
  \subfloat[]{\includegraphics[width=0.495\textwidth]{../images/dark_meson/crossSections_su2l_gaugephobic.pdf}\phantomsubcaption\label{fig:crosssections_su2l}}
  \subfloat[]{\includegraphics[width=0.495\textwidth]{../images/dark_meson/crossSections_su2r_gaugephobic.pdf}\phantomsubcaption\label{fig:crosssections_su2r}}
  \caption{Pair-production cross-sections for dark pions as a function of dark pion mass for four different values of $\etaD$ in \protect\subref{fig:crosssections_su2l} an \sul model and \protect\subref{fig:crosssections_su2r} an \sur model. The dashed colored lines indicate the contribution of the resonant production mode to the total dark pion production cross-section.}
  \label{fig:crossections}
\end{figure}

\begin{figure}[tbp]
  % The plots have been produced with the plot_crossSections.C script which can be found in the dark meson git space: https://gitlab.cern.ch/darkmesonsearch/darkscripts/blob/master/plot_crossSections.C
  \centering
  \subfloat[]{\includegraphics[page=3,width=0.495\textwidth]{../images/dark_meson/branchingFractions.pdf}\phantomsubcaption\label{fig:pi0branchingfraction}}
  \subfloat[]{\includegraphics[page=4,width=0.495\textwidth]{../images/dark_meson/branchingFractions.pdf}\phantomsubcaption\label{fig:pipbranchingfraction}}
  %\includegraphics[width=0.32\textwidth]{crossSections_su2r_gaugephobic.pdf}
  \caption{Branching fractions of the available decays of dark pions from gaugephobic \sul and \sur  models are shown for \protect\subref{fig:pi0branchingfraction} neutral dark pions and \protect\subref{fig:pipbranchingfraction} positively charged dark pions. Channels with small branching fractions are suppressed for clarity.}
  \label{fig:branchingfractions}
\end{figure}

This search is the result of the analysis of \lumi of proton–proton (\pp) collisions collected by the ATLAS detector during Run 2 of the Large Hadron Collider (LHC).  Since the dark pions are pair-produced in the model considered, the experimental signatures are either three top quarks and one bottom quark ($tttb$) or two top quarks and two bottom quarks ($ttbb$)\footnote{The label "$tttb$" is used to indicate both $t\bar{t}t\bar{b}$ as well as its charge conjugate, $t\bar{t}\bar{t}b$; whereas "$ttbb$" refers to $t\bar{t}b\bar{b}$.}. About one third of dark pions in the \sul models are neutral, resulting in the $tttb$ event signature being twice as likely as the $ttbb$ signature. These processes can give rise to several different final states depending on the hadronic
or semileptonic decay mode of each of the top quarks. 
The search is performed in the \emph{\allhad} channel, targeting fully hadronic top quark decays where the signal results in eight to ten jets of which at least four originate from bottom quarks, and in the \emph{\onelep} channel, corresponding to final states with exactly one electron or muon in addition to up to four jets from b-quarks.

The results are interpreted as limits on the production cross-section of dark pion pairs as a function of $m_{\piD}$ and \etaD.


