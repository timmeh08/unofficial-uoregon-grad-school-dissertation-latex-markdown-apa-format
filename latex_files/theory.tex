It cannot be overstated that humanity is driven to understand the nature of the universe, leading to great efforts and achievements in theoretical and experimental physics.
One of these great achievements, a combined effort of many across decades and through multiple experiments, is the formulation of a framework to describe the fundamental particles and their interactions.
This framework is known as the Standard Model (SM) of particle physics.
Complete in its ability to describe and predict the phenomena of the observed particles it contains, it remains the starting point for virtually every investigation in particle physics.
The most recent confirmations of the SM are the top quark observation at Fermilab in 1995 by the D0 and CDF experiments \cite{PhysRevLett.74.2626,PhysRevLett.74.2632} and the Higgs boson discovery at CERN in 2012 by the ATLAS and CMS experiments \cite{HIGG-2012-27,CMS-HIG-12-028}.


Despite its successes, however, the Standard Model is known to be incomplete in regards to several observations.
We know that over 85\% of the matter in the observable universe is in the form of dark matter (DM), which does not have any observed interactions with the SM and thus must be explained with new physics.
Additionally, the SM requires neutrinos to be massless, which is condraticted by neutrino oscillation observations, and the extensions needed to explain neutrino masses are beyond the scope of this thesis.
The matter-antimatter asymmetry observed in the universe is also not fully explained by the SM, requiring some new mechanism to generate the observed abundance of matter.
A strongly-coupled theory of quantum gravity is unable to be incorporated into the SM, and there is no explanation within the SM for dark energy at the scales observed in cosmology.
These shortcomings and others motivate searches for physics beyond the Standard Model (BSM).
This chapter will discuss the Standard Model briefly, with emphasis on electroweak symmetry breaking and the Higgs mechanism, before discussing the motivation for BSM physics and two possible dark sectors in more detail.

\section{A Brief Overview of the Standard Model}

The Standard Model is a quantum field theory (QFT) describing known particles and three of the four fundamental forces of nature, the strong, electromagentic (EM), and weak forces through fermion fields interacting via gauge boson fields.
These fields are colloquially referred to as particles, with the fermions as the matter particles and the gauge bosons as the force carriers.
Fermion field masses are generated through interactions with the Higgs field via the Higgs mechanism, and the broken electroweak symmetry gives mass to the weak gauge bosons as well.
All SM particles are visualized in Fig. \ref{sm_particles}.

\begin{figure}[htbp]
    \centering
    \includegraphics[width=0.8\textwidth]{../images/my_images/sm_particles.png}
    \caption{The Standard Model particles \cite{Dominguez:2002395}. The three generations of fermions are shown at left, the Higgs boson at the center, and the gauge bosons at right.}
    \label{sm_particles}
\end{figure}