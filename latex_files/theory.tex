It cannot be overstated that humanity is driven to understand the nature of the universe, leading to great efforts and achievements in theoretical and experimental physics.
One of these great achievements, a combined effort of many across decades and through multiple experiments, is the formulation of a framework to describe the fundamental particles and their interactions.
This framework is known as the Standard Model (SM) of particle physics.
Complete in its ability to describe and predict the phenomena of the observed particles it contains, it remains the starting point for virtually every investigation in particle physics.
The most recent confirmations of the SM are the top quark observation at Fermilab in 1995 by the D0 and CDF experiments \cite{PhysRevLett.74.2626,PhysRevLett.74.2632} and the Higgs boson discovery at CERN in 2012 by the ATLAS and CMS experiments \cite{HIGG-2012-27,CMS-HIG-12-028}.


Despite its successes, however, the Standard Model is known to be incomplete in regards to several observations.
We know that over 85\% of the matter in the observable universe is in the form of dark matter (DM), which does not have any observed interactions with the SM and thus must be explained with new physics.
Additionally, the SM requires neutrinos to be massless, which is condraticted by neutrino oscillation observations, and the extensions needed to explain neutrino masses are beyond the scope of this thesis.
The matter-antimatter asymmetry observed in the universe is also not fully explained by the SM, requiring some new mechanism to generate the observed abundance of matter.
A strongly-coupled theory of quantum gravity is unable to be incorporated into the SM, and there is no explanation within the SM for dark energy at the scales observed in cosmology.
These shortcomings and others motivate searches for physics beyond the Standard Model (BSM).
This chapter will discuss the Standard Model briefly, with emphasis on electroweak symmetry breaking and the Higgs mechanism, before discussing the motivation for BSM physics and two possible dark sectors in more detail.

\section{A Brief Overview of the Standard Model}

The Standard Model is a quantum field theory (QFT) describing known particles and three of the four fundamental forces of nature, the strong, electromagentic (EM), and weak forces through fermion fields interacting via gauge boson fields.
These fields are colloquially referred to as particles, with the fermions as the matter particles and the gauge bosons as the force carriers.
Fermion field masses are generated through interactions with the Higgs field via the Higgs mechanism, and the broken electroweak symmetry gives mass to the weak gauge bosons as well.
The fermions are spin 1/2 particles, the force carriers are spin 1 particles, and the Higgs boson is a spin 0 particle.
Fermions are further classified as quarks and leptons, with quarks charged under both the strong and electroweak interactions, while leptons only interact via the electroweak interaction.
All SM particles are visualized in Fig. \ref{sm_particles}.

\begin{figure}[htbp]
    \centering
    \includegraphics[width=0.8\textwidth]{../images/my_images/sm_particles.png}
    \caption{The Standard Model particles \cite{Dominguez:2002395}. The three generations of fermions are shown at left, the Higgs boson at the center, and the gauge bosons at right.}
    \label{sm_particles}
\end{figure}

As a gauge theory, the SM is invariant under the symmetry group $SU(3)_C \times SU(2)_L \times U(1)_Y$.
$SU(3)_C$ describes the strong interaction with color charge $C$, and $SU(2)_L \times U(1)_Y$ describes the electroweak interaction with weak isospin $L$, since weak charge currents only interact with left-handed fermions, and hypercharge $Y$ acting on right and left-handed particles to give the observed electric charge of particles.
Table \ref{sm_charges} summarizes the gauge quantum numbers of the SM fermions under these symmetry groups.

\begin{table}[htbp]
    \centering
    \begin{tabular}{|c|c|c|c|}
        \hline
        Particle & $SU(3)_C$ & $SU(2)_L$ & $U(1)_Y$ \\
        \hline
        Left-handed quark doublet $Q_L$ & 3 & 2 & 1/6 \\
        Right-handed up quark $u_R$ & 3 & 1 & 2/3 \\
        Right-handed down quark $d_R$ & 3 & 1 & -1/3 \\
        Left-handed lepton doublet $L_L$ & 1 & 2 & -1/2 \\
        Right-handed electron $e_R$ & 1 & 1 & -1 \\
        \hline
    \end{tabular}
    \caption{Gauge quantum numbers of the Standard Model fermion fields. The three fermion generations are not shown, as all generations have the same quantum numbers.}
    \label{sm_charges}
\end{table}


\subsection{Electroweak Symmetry Breaking and the Higgs Mechanism}

In the SM, the electroweak symmetry $SU(2)_L \times U(1)_Y$ is unbroken at high energies.
The gauge fields associated with this symmetry are the $W^1$, $W^2$, and $W^3$ bosons for $SU(2)_L$ and the $B$ boson for $U(1)_Y$.
However, at low energies, the Higgs field spontaneously breaks this symmetry to the electromagnetic symmetry $U(1)_{EM}$, giving mass to the weak gauge bosons and fermions.
This is accomplished through the introduction of a complex scalar doublet field $\phi$ into the SM

\begin{equation}
    V(\phi) = \frac{1}{2}\mu^2 \phi^\dagger \phi + \frac{1}{4}\lambda (\phi^\dagger \phi)^2,
    \label{higgs_potential}
\end{equation}

which has stationary points at $\phi^\dagger \phi = 0, \pm\mu^2/\lambda$.
For $\mu^2 < 0$ and $\lambda > 0$, the minimum at 0 is unstable.
Even if the field is initially at 0, the field would evolve into the non-zero minima, spontaneously breaking the electroweak symmetry since all non-zero minima are degenerate.
Choosing $v = \sqrt{-\mu^2/\lambda}$ as the vacuum expectation value (VEV) of the Higgs field, we can write the Higgs field in unitary gauge as

\begin{equation}
    \phi = \frac{1}{\sqrt{2}} \begin{pmatrix} 0 \\ v + H(x) \end{pmatrix},
    \label{higgs_field}
\end{equation}

With $v \approx 246$ GeV and $H(x)$ the physical Higgs boson field.
Plugging Eq. \ref{higgs_field} into the SM Lagrangian gives mass terms for the weak gauge bosons and fermions, with the photon remaining massless, as well as a mass term for the Higgs boson itself.
The $W^1$, $W^2$, and $W^3$ bosons mix with the $B$ boson to give the physical $W^{\pm}$, $Z$, and $\gamma$ bosons.
The masses of the weak gauge bosons are given by 
\begin{equation}
    m_W = \frac{1}{2}vg, \quad m_Z = \frac{1}{2}v\sqrt{g^2 + g'^2},
    \label{weak_boson_masses}
\end{equation}
where $g$ and $g'$ are the $SU(2)_L$ and $U(1)_Y$ coupling constants, respectively.
Fermion masses are generated through Yukawa interactions with the Higgs field, with the mass of a fermion $f$ given by 
\begin{equation}
    m_f = \frac{y_f v}{\sqrt{2}},
    \label{fermion_masses}
\end{equation}
where $y_f$ is the Yukawa coupling of the fermion to the Higgs field.
With the Higgs field incorporated into the SM, tri and quartic self-interactions of the Higgs boson are also generated, with the Higgs boson mass given by 
\begin{equation}
    m_H = \sqrt{2\lambda} v.
    \label{higgs_mass}
\end{equation}
The discovery of the Higgs boson at the LHC with a mass of approximately 125 GeV \cite{HIGG-2012-27,CMS-HIG-12-028} confirms a viable candidate for the SM Higgs boson.
However, there are still many properties of the Higgs and electroweak symmetry breaking that remain to be measured and understood, including the Higgs self-couplings and possible couplings to BSM particles.

\section{Evidence for Dark Matter and Beyond the Standard Model Physics}

As mentioned previously, there are several observations that the Standard Model is unable to explain, motivating searches for physics beyond the Standard Model.
One of the most compelling pieces of evidence for BSM physics is the existence of dark matter. 
Astrophysical and cosmological observations provide strong evidence through the gravitational effects of dark matter on visible matter.
This includes galaxy rotation curves \cite{1980ApJ...238..471R}, gravitational lensing \cite{Massey_2010}, and the cosmic microwave background (CMB) anisotropies \cite{Hu_2002}.
In particular, the rotation curve of galaxy M33, where the observed rotation curve does not match the expectation due to visible matter, indicates the presence of additional unseen mass, attributed to dark matter.
The observation of the Bullet Cluster, where two galaxy clusters have collided and the dark matter component has been observed to separate from the baryonic matter, provides a clear picture of graviational lensing.
Both examples can be seen in Fig. \ref{galaxy_rotation_curve}.

\begin{figure}[htbp]
    \centering
    \includegraphics[width=0.45\textwidth]{../images/my_images/astronomy-02-00007-g001.png}
    \includegraphics[width=0.45\textwidth]{../images/my_images/bullet_cluster.png}
    \caption{The observed rotation curve of galaxy M33 compared to the expected rotation curve from visible matter \cite{astronomy2020007} (left) and the Bullet Cluster showing the separation of dark matter and baryonic matter through gravitational lensing \cite{Massey_2010} (right).}
    \label{galaxy_rotation_curve}
\end{figure}

Current attempts to explain these observations through modifications to gravity have not been successful in explaining all observed phenomena, leading to the conclusion that dark matter is most likely composed of new, yet-undiscovered particles.
In addition to observational evidence, large scale structure formation and simulations of galaxy formation also require the presence of dark matter to match observations \cite{2020NatRP...2...42V}.
While the evidence is strong, the exact physical nature of dark matter remains unknown as of now.
Dark matter must meet several criteria from current observational constraints, however, including being stable on cosmological timescales, non-baryonic, having weak or no interactions with electromagnetic radiation, and the correct relic abundance to account for the observed dark matter density in the universe.
These criteria have led to many proposed candidates for dark matter particles, including weakly interacting massive particles (WIMPs), axions, and sterile neutrinos, among others.
It is possible that dark matter might be contained within a dark sector, with its own particles and interactions, accessible to the Standard Model only through so-called portal interactions.
The Higgs boson is a prime candidate for such a portal, given its unique properties, coupling to mass, and the current lack of constraints on the Higgs ito invisible branching ratio.
In order to explore these possiblities experimentally, there must be some particle(s) that mediate the interactions between the dark sector and the Standard Model, and have an accessible mass scale for current experiments such as the LHC.
This section will discuss two possible dark sector models that could be probed through Higgs portal interactions at the LHC: the dark meson sector of stealth dark matter and a dark photon model.

\subsection{Stealth Dark Matter and Dark Mesons}

Dark matter has eluded direct detection thus far, with increasingly stringent limits being placed on the interaction cross section of dark matter with Standard Model particles.
This has led to the creation and exploration of models where the absence of an experimental signature so far is a natural consequence of the model itself.
One such area of dark sector models is stealth dark matter \cite{Kribs_2016}, which posits a new confining non-Abelian gauge group, similar to quantum chromodynamics (QCD), with dark fermions that are charged under the dark gauge group and transform as vector-like representations of the SM electroweak group.

The specific dark meson model explored in this thesis is detailed in Ref. \cite{kribs_2019}.
The dark matter candidate in this model is a composite baryonic scalar of an $SU(N)_D$ strongly interacting dark sector, with $N$ being even.
The lightest baryonic scalar is stable due to an accidental $U(1)_{dark baryon}$ symmetry of the dark sector, making it a viable dark matter candidate.
The dark fermions also form mesonic bound states, called dark mesons, that could be accessible at the LHC.
These dark mesons are previously unconstrained by existing searches due to the nature of BSM searches that typically require high MET or high mass ressonances while the dark meson mass regime is lower.

This new dark sector contains at least one triplet of dark pions and dark rhos, $(\pi^{\pm}_D,\pi^0_D)$ and $(\rho^{\pm}_D,\rho^0_D)$, which are pseudoscalars and vectors respectively.
Charged under the electroweak interaction and $SU(N)_D$, the dark fermions break the global chiral symmetry of the dark sector, leading to the decay of the dark pions and rhos to Standard Model particles through electroweak interactions. 
The dark meson model has a custodial Higgs symmetry with SU(2) dark flavor symmetry, which protects the electroweak $T$ parameter from large corrections.
The Lagrangian is thus

\begin{equation}
    \mathcal{L}_{dark sector} = \mathcal{L}_{kinetic} + \mathcal{L}_{mass} + \mathcal{L}_{\pi_D,\rho_D} + \mathcal{L}_{kinetic mixing} + \mathcal{L}_{decay},
    \label{dark_sector_lagrangian}
\end{equation}

and includes kinetic terms, mass terms, dark meson interaction terms, kinetic mixing terms between the dark sector and SM gauge bosons, and decay terms for the dark mesons to SM particles.

Searches for dark mesons at the LHC can be parameterized by the dark pion mass $m_{\pi_D}$, along with the ratio of the dark pion mass to the dark rho mass,

\begin{equation}
    \eta = \frac{m_{\pi_D}}{m_{\rho_D}}
    \label{eta_definition}
\end{equation}

If $\eta$ < .5, the dark rho can decay to two dark pions with a branching fraction near 1, while if $\eta$ > .5, the dark rho decays to pairs of SM fermions.
The electroweak portal between the dark sector and SM is primarily through the kinetic mixing term,

\begin{equation}
    \mathcal{L}_{kinetic mixing} = \frac{\epsilon}{2} \rho^{a}_{D,\mu\nu} F^{a\mu\nu}.
    \label{kinetic_mixing}
\end{equation}

Two mixing models are allowed, SU(2)$_R$ and SU(2)$_L$, with the former having the dark rho mix with the full SM $W$ boson triplet and the latter having the dark rho mix with the B field only.
The SU(2)$_L$ model has a significantly higher cross section, and is thus the focus of Chapter \ref{ch:dark_mesons}.
Fig. \ref{dark_meson_feynman} shows the relevant Feynman diagrams for dark meson production at leading order, both through kinetic mixing and via Drell-Yan production.
The resonant production via kinetic mixing is the dominant production mode, with the dark pions decaying to SM top and bottom quarks.

\begin{figure}[htbp]
    \centering
    \includegraphics[width=0.8\textwidth]{../images/my_images/dark_meson_feynman.png}
    \caption{Dark meson production at leading order. The dominant production mechanism through kinetic mixing is shown on the left and middle, while Drell-Yan pair production is shown on the right.}
    \label{dark_meson_feynman}
\end{figure}

\subsection{The Dark Photon}

Another more minimal extension to the Standard Model to access a dark sector that has gained traction in recent years is through the introduction of a new $U(1)_D$ gauge symmetry, with the associated gauge boson being the dark photon, $\gamma_D$.
The dark photon can either be massless or massive, with the former being the focus of this thesis.
For a massive dark photon, kinetic mixing between the dark photon and the SM hypercharge gauge boson $B$ is allowed.
A massless dark photon cannot couple at tree level to any SM particles, and interacts via messenger fields of dimension higher than 4.
The generic Lagrangion for the SM U(1) and the dark U(1)$_D$ is given by
\begin{equation}
    \mathcal{L} = -\frac{1}{4} F_{\mu\nu} F^{\mu\nu} - \frac{1}{4} F'_{\mu\nu} F'^{\mu\nu} + \frac{\epsilon}{2} F_{\mu\nu} F'^{\mu\nu} + \frac{1}{2} m^2_{\gamma_D} A'_\mu A'^\mu + J^\mu_{EM} A_\mu + J^\mu_{dark} A'_\mu,
    \label{dark_photon_lagrangian}
\end{equation}

where $F_{\mu\nu}$ and $F'_{\mu\nu}$ are the field strength tensors for the SM photon and dark photon, respectively, $\epsilon$ is the kinetic mixing parameter, $m_{\gamma_D}$ is the mass of the dark photon, and $J^\mu_{EM}$ and $J^\mu_{dark}$ are the electromagnetic and dark currents, respectively.
Assuming $m_{\gamma_D}$ = 0, we can cancel the kinetic mixing term through a field redefinition, leaving the dark photon with no tree-level couplings to SM particles.
This redifinition is done by defining new fields $A_\mu$ and $A'_\mu$ as follows:

\begin{equation}
    \begin{pmatrix} A'^\mu \\ A^\mu \end{pmatrix} = \begin{pmatrix} \frac{1}{\sqrt{1-\epsilon^2}} & 0 \\ -\frac{\epsilon}{\sqrt{1-\epsilon^2}} & 1 \end{pmatrix} \begin{pmatrix} \cos\theta & -\sin\theta \\ \sin\theta & \cos\theta \end{pmatrix} \begin{pmatrix} \tilde{A}^\mu_d \\ \tilde{A}^\mu \end{pmatrix},
    \label{field_redefinition}
\end{equation}

where $\tilde{A}^\mu_d$ and $\tilde{A}^\mu$ are the new fields after the redefinition, and $\theta$ is a mixing angle that can diagonolize the kinetic terms.
Since $\theta$ can take any arbitry angle without a dark photon mass term, interactions between the dark photon and SM particles only happen through higher order operators.

The dark photon model of interest in \ref{ch:dark_photons} is a minimal model that assumes a generic messenger sector that allows for interactions between the dark photon and the SM through the afformentioned higher order operators \cite{Biswas_2022}.
The doublet messenger fields, denoted by $S_i$, carry both SM and dark U(1)$_D$ charges, and generate an interaction between the dark photon and the SM at the one loop level.
An interaction Lagrangian for this model can be written as

\begin{equation}
    \mathcal{L}_{int} = \lambda_S S_0 (\tilde{H}^\dagger S^{U}_{L} S^{U}_{R} + H^\dagger S^{D}_{L} S^{D}_{R}) + h.c.
    \label{dark_photon_interaction_lagrangian}
\end{equation}

where $\lambda$ is a universal coupling, the $S_0$ is a singlet scalar field with a vev, and $S^{U}_{L,R}$ and $S^{D}_{L,R}$ are the doublet up-type and down-type messenger fields, respectively.
Once the scalar field $S_0$ acquires a vev, the messenger fields acquire mass and generate trilinear couplings between the dark photon and the SM Higgs boson at one loop level, $H_{\gamma\gamma_D}$ and $H_{\gamma_D\gamma_D}$, with couplings proportional to $\mu_S=\lambda_S<S_0>$.
After electroweak symmetry breaking, a mixing term between the left and right messenger fields is generated, proportional to $\mu_S\nu$ where $\nu$ is the Higgs vev.
Feynman diagrams that contribute to the $H_{\gamma\gamma_D}$ coupling are shown in Fig. \ref{dark_photon_feynman}.

\begin{figure}[htbp]
    \centering
    \includegraphics[width=0.8\textwidth]{../images/my_images/dark_photon_feynman.png}
    \caption{Feynman diagrams that contribute to the $H_{\gamma\gamma_D}$ coupling at one loop level. The dark photon is denoted by $\gamma_D$, the Higgs boson by $H$, and the messenger fields by $S$.}
    \label{dark_photon_feynman}
\end{figure}
