The ATLAS detector is a general purpose detector at Point 1 of the LHC ring.
It is designed to detect and measure the scattering products of collisions provided by the LHC.
Cylindrical in design, it is symmetric about the z-axis, and measures approximately 25 m in height and 44 m in length, with a total mass of about 7000 tons. \cite{PERF-2007-01}.
ATLAS consists of several detector subsystems, magnets, and cryostats arranged in layers around the interaction point, as shown in Fig. \ref{atlas_schematic}.
The first detector layer surrounding the interaction point is the Inner Detector (ID) which tracks charged particles.
Around the ID are multiple calorimeter systems which measure and absorb the energy of all Standard Model particles except muons and neutrinos.
The final detector component is the Muon Spectrometer (MS) which tracks muons as they pass through ATLAS.
The detector systems are divided into barrel and end cap regions, with the barrel covering the region perpendicular to the beamline and the end caps covering regions parallel to the beamline.
This design allows for near 4$\pi$ coverage around the interaction point, and is sensitive to a wide range of particle signatures, motivated partly by the prediction that the Higgs boson would decay to multiple final states.
There are two main magnet systems in ATLAS: a solenoid magnet surrounding the ID, and a toroidal magnet surrounding the calorimeters and sections of the MS.
Bathing sections of the detector in strong magnetic fields allows for the measurement of charged particle momenta by curving their trajectories.
A vital off-detector system is the Trigger and Data Acquisition (TDAQ) system, which takes input from the detector components and selects interesting collision events for further analysis. 

\begin{figure}[htbp]
    \centering
    \includegraphics[width=0.9\textwidth]{../images/my_images/atlas_schematic.jpg}
    \caption{Schematic diagram of the ATLAS detector and its sub-systems \cite{PERF-2007-01}.}
    \label{atlas_schematic}
\end{figure}

\subsection{The ATLAS Coordinate System}\label{subsec:atlas_coordinates}
ATLAS uses a right-handed coordinate system with its origin at the interaction point, $\hat{y}$ pointed upwards, $\hat{x}$ pointed radially inward towards the center of the LHC ring, and $\hat{z}$ pointed along the beamline.
The cylindrical symmetry of the detector and typical particle trajectories often motivates the use of two angular coordinates with $\theta$ the angle from the $\hat{z}$ axis, and $\phi$ the azimuthal angle in the transverse (x-y) plane.

To better describe particle trajectories in a flat geometry, the pseudorapidity $\eta$ replaces the polar angle $\theta$ and is defined as:

\begin{equation}\label{pseudorapidity}
    \eta = -\ln\left(\tan\left(\frac{\theta}{2}\right)\right)
\end{equation}

Pseudorapidity is valid in the highly relativistic conditions of the LHC where $pc \gg mc^2$, and has the advantage of being invariant under Lorentz boosts along the beam axis.
A common measure of distance in the detector, since it is also invariant under boosts along the beam axis, is $\Delta R$ defined as:

\begin{equation}\label{deltaR}
    \Delta R = \sqrt{(\Delta \eta)^2 + (\Delta \phi)^2}
\end{equation} 

Since the LHC collides protons essentially head-on, they have an initial transverse energy of zero.
It is therefore useful to define and measure physical quantities in the transverse plane, such as transverse momentum $p_T$ and missing transverse energy $E_T^{miss}$ often referred to as MET.

\subsection{The Inner Detector}\label{subsec:inner_detector}

The ID is the innermost detector system seen in Fig. \ref{ID}, tracking charged particle trajectories as they emerge from the interaction point with minimal energy loss as they pass through \cite{ATLAS-TDR-04}.
It is composed of three subsystems and covers an $\eta$ range of $|\eta| < 2.5$.
The whole ID is immersed in a 2T magnetic field provided by the solenoid magnet.

Tracking is performed by measuring the charge, position, and timing of charged particles.
As a charged particle passes through a layer of the ID, it ionizes the detector material, creating electron-hole pairs.
Each layer is under high voltage, and these free charges are collected to provide a \textit{hit} in the detector.
Combining hits from multiple layers allows for the reconstruction of the particle trajectory, or \textit{track}.
The direction of curvature of the track due to the magnetic field gives the charge of the particle, while the radius of curvature gives the momentum.
Tracks are also traced back to the beamline to identify the vertex of the associated particle. 
If a track does not originate from the beamline, it might have originated from a secondary vertex, indicating the decay of a long-lived particle or as a product of a parent particle decay.

\begin{figure}[htbp]
    \centering
    \includegraphics[width=0.7\textwidth]{../images/my_images/IDbriefing_figure1.png}
    \caption{Diagram of the ATLAS Inner Detector and its three subsystems \cite{PERF-2007-01}.}
    \label{ID}
\end{figure}

The innermost subsystem is the Pixel Detector \cite{ATLAS-TDR-11}, starting at a distance of 33.25 mm from the beamline and extending to 122.5 mm.
It is composed of layers of 92 million silicon pixel sensors, each measuring 50 $\mu$m by 400-600 $\mu$m in size.
Sensors are stacked in four overlapping layers in the barrel region, and 3 disks on each end cap providing coverage out to $|\eta| < 2.5$.
The overlapping geometry provides a granularity of approximately 10 $\mu$m.
The closest layer to the beamline is the Insertable B-Layer (IBL) added in 2014 for Run 2 operation and beyond \cite{ATLAS-TDR-19}.
The purpose of the IBL upgrade is to improve the resolution of secondary vertices and thus b-tagging performance, as well as further suppress pileup vertices.

Surrounding the Pixel Detector is the Semiconductor Tracker (SCT) \cite{IDET-2013-01} at a radial distance of 122.5 mm to 520 mm, composed of silicon microstrip sensors.
SCT sensors are arranged in pairs of back-to-back strips and are larger than pixel sensors.
There are four SCT barrel layers out to $\eta$ < 1, and nine end cap disks on each side extending coverage to $|\eta| < 2.5$.
The precision of the SCT is approximately 16 $\mu$m in the r-$\phi$ direction and 580 $\mu$m in the z direction.

The outermost subsystem of the ID is the Transition Radiation Tracker (TRT) \cite{PERF-2007-01}, extending from a radial distance of 563 mm to 1066 mm.
It uses a completely different technology than the inner two systems, consisting of 298,304 drift tubes, or straws.
Each straw is a 4 mm diameter tube filled with an ionizing gas mixture and a central grounded wire made of gold plated tungsten.
The gas mixture is 70\% Xe, 27\% CO$_2$, and 3\% O$_2$.
Extending out to $|\eta| < 2.0$, the TRT has a barrel section with straws parallel to the beamline, and two end cap sections with straws arranged radially, and provides ~36 additional hits to improve tracking.
As particles pass through the polypropylene material between straws, they emit transition radiation, which ionizes the gas in the straws.
The larger size of the TRT provides a better measurement of the bend of tracks in the magnetic field, improving momentum resolution.

\subsection{The Calorimeter System}\label{subsec:calorimeters}

The purpose of the ATLAS calorimeter system is to absorb and measure the energy of particles after they pass through the ID.
There are two main sampling calorimeter technologies used in ATLAS, Liquid Argon (LAr) and scintillating tiles (Tile), composing multiple layers as shown in Fig. \ref{calorimeters}.
LAr calorimeters are used in the electromagnetic (EM) barrel, hadronic and EM end caps (HEC and EMEC), and the forward calorimeter (FCAL).
The Tile barrel and extended barrel use scintillating tiles, and are for hadronic object reconstruction.
With over 188,000 readout channels, the calorimeter system covers the range $|\eta| < 4.9$ \cite{ATLAS-TDR-02}.
The calorimeters were designed to fully contain particle showers up to a few TeV in energy and provide identification of electrons and photons.
Hadronic showering primarily occurs in the Tile systems, but jet reconstruction takes input from all calorimeters.
Muons and neutrinos are the only particles that escape the calorimeters with minimal energy loss, however, muons are tracked in the MS and neutrinos can be inferred from the missing transverse energy in an event.

\begin{figure}[htbp]
    \centering
    \includegraphics[width=0.7\textwidth]{../images/my_images/calorimeter.jpg}
    \caption{Diagram of the ATLAS calorimeter system labeled by subsystem \cite{PERF-2007-01}.}
    \label{calorimeters}
\end{figure}

Sampling calorimetry involves alternating layers of dense absorbing material and active sensing material.
Absorbing layers induce particle showers by scattering the incoming particle(s).
The particle showers interact with the active layers through ionization or scintillation, allowing for multiple measurements of energy deposition as they pass through the calorimeter.
The calorimter resolution within ATLAS is generally parameterized as:

\begin{equation}\label{calorimeter_resolution}
    \frac{\sigma_E}{E} = \frac{a}{\sqrt{E}} \oplus \frac{b}{E} \oplus c
\end{equation}

and depends on stochastic effects ($a$), electronic noise ($b$), and other detector effects ($c$).
In practice, resolution varies between calorimeter subsystems, but typical values for jets range from 6-25\% for 20 GeV $< E <$ 300 GeV \cite{JETM-2018-05}.  

\subsubsection{Electromagnetic Calorimeters}\label{subsubsec:em_calorimeters}

The ATLAS EM calorimeters \cite{ATLAS-TDR-02} primarily measure the energy of electrons and photons, and are sectioned into a barrel and two end cap regions, extending out to a radius of 2 m.
All regions use LAr as the active material and lead as the absorber, chosen for its high Z value, with radiation lengths of ~24 $X_0$ in the barrel and ~26 $X_0$ in the end caps.
The barrel covers the region $|\eta| < 1.475$, while the two end cap sections cover $1.375 < |\eta| < 3.2$.
An accordion geometry is used for the lead absorbers as shown in Fig. \ref{em_calorimeter}, with LAr filling the gaps between sheets. 
Electrodes are placed in the LAr gaps and are sectioned into "cells" in $\eta$ and $\phi$ to provide highly granular spatial resolution.
This design allows for full $\eta$-$\phi$ coverage and fast signal collection.

\begin{figure}[htbp]
    \centering
    \includegraphics[width=0.45\textwidth]{../images/my_images/accordion.png}
    \includegraphics[width=0.45\textwidth]{../images/my_images/lar_cells.png}
    \caption{Accordion geometry of the ATLAS electromagnetic calorimeter with readout schematic (left) \cite{nikiforou2013performanceatlasliquidargon}. EM barrel layout with cells showing coverage in $\eta$ and $\phi$ (right) \cite{ATLAS-TDR-02}.}
    \label{em_calorimeter}
\end{figure}

There are three layers in the barrel region, with an additional presampler in front of the barrel to allow for energy loss corrections due to material in front of the calorimeter.
As electrons and photons pass through the layers, they interact with the lead absorbers, producing a shower of more electrons and photons primarily through Bremsstrahlung and pair production.
This cascade continues, with measurements and showering occuring in turns until all the energy is absorbed.
The first layer has very fine granularity in $\eta$-$\phi$, 0.003 $\times$ 0.1, to distinguish between single photons and overlapping photons from $\pi^0$ decays.
Most of the energy is deposited in the second layer as it is the thickest, with a trade off between slightly lower $\eta$ granularity than layer one, but higher $\phi$ granularity at 0.025 $\times$ 0.025.
The third layer is thinner and collects the tail ends of showers with a granularity of 0.050 $\times$ 0.025.
The EM end caps have a similar accordion geometry and layer structure, with an additional presampler between the barrel and end caps, but are arranged in a disk shape to cover the forward regions.
EMEC cells are segmented in cells that have varying granularity with $\eta$ and layer and are larger than barrel cells.
There are over 170,000 readout channels in the EM calorimeters.

The readout of the EM calorimeters is delayed by the ion drift time in the LAr, which varies between 450-600 ns depending on the layer \cite{LARG-2009-02}.
This delayed signal is longer than the LHC collision rate of 24 ns, necessitating the use of pulse shaping to mitigate out-of-time pileup from previous bunch crossings.
Analog circuits in the front end electronics use a bipolar shaping filter to create a sharply peaked pulse and long negative tail as shown in Fig. \ref{pulse_shaping}.
The shaped pulse has a total integral of zero, canceling out contributions from out-of-time pileup.

\begin{figure}[htbp]
    \centering
    \includegraphics[width=0.6\textwidth]{../images/my_images/pulse_shape.png}
    \caption{Pulse shape of LAr calorimeter signal to mitigate out-of-time pileup \cite{ATLAS-TDR-22}.}
    \label{pulse_shaping}
\end{figure}

\subsubsection{Hadronic Calorimeters}\label{subsubsec:hadronic_calorimeters}

The physically largest component of the ATLAS calorimeter system is the hadronic calorimeters, which capture and measure the energy of hadrons.
Particles such as charged and neutral pions, kaons, and protons punch through the EM calorimeters and interact via the strong interaction with nuclei in the absorber material, producing hadronic showers.
These showers are commonly described through the interaction length $\lambda$, where a hadron has a $1-\frac{1}{e}$ chance of interacting with a nucleus.
Composed of the Tile calorimeter in the range $|\eta| < 1.475$, and the HEC in $1.475 < |\eta| < 3.2$, scintillators and LAr are used respectively.
Interaction lengths extend out to ~17 in Tile and ~11 in the HEC.

The Tile calorimeter \cite{ATLAS-TDR-03} uses steel as the absorber and plastic scintillating tiles as the active material as shown in Fig. \ref{tile_calorimeter}, with a radius from 2.28 m to 4.25 m.
It takes its name from the rectangular plastic scintillating tiles that are arranged in layers between steel absorber plates.
When shower particles passes through the tiles, they excite molecules in the plastic, which in turn emit photons as they return to their ground state.
Photons are then collected by wavelength shifting fibers and directed to photomultioplier tubes (PMTs) which convert the light to a voltage signal, and is proportional to the energy deposited in the tile.
Tiles are grouped together by the PMTs into cells with coarser granularity than the EM calorimeters due to the larger size of hadronic showers, with $\eta \times \phi$ sizes of 0.1 $\times$ 0.1 in the barrel and 0.1 $\times$ 0.2 in the extended barrel.

\begin{figure}[htbp]
    \centering
    \includegraphics[width=0.5\textwidth]{../images/my_images/tile_calorimeter.png}
    \caption{Tile calorimeter wedge module showing the scintillating tiles, steel absorber plates, and PMTs \cite{ATLAS-TDR-28}.}
    \label{tile_calorimeter}
\end{figure}

Just past the EMEC in $z$ is the HEC, which is similar in design to the EM calorimeters, using LAr as the active material and copper as the absorber \cite{ATLAS-TDR-02} arranged in the accordion geometry.
As seen in Fig. \ref{hec_calorimeter}, each end cap has two wheels contained in the same cryostat as the EMEC and FCAL.
The cells also vary in size depending on $\eta$ and layer due to the wheel geometry.

\begin{figure}[htbp]
    \centering
    \includegraphics[width=0.6\textwidth]{../images/my_images/hec.png}
    \caption{Diagram of the ATLAS hadronic end cap calorimeter along with the EMEC and FCAL \cite{ATLAS-TDR-02}.}
    \label{hec_calorimeter}
\end{figure}

\subsection{The Muon Spectrometer}\label{subsec:muon_spectrometer}

\begin{figure}[htbp]
    \centering
    \includegraphics[width=0.7\textwidth]{../images/my_images/muon_spec.png}
    \caption{Diagram of the ATLAS Muon Spectrometer chambers \cite{PERF-2007-01}.}
    \label{muon_spectrometer}
\end{figure}

The outermost detector subsystem is the Muon Spectrometer (MS) \cite{ATLAS-TDR-10}, which tracks muons since they pass through ATLAS with minimal energy loss, and measures their momenta.
There are multiple layers of tracking chambers using different technologies with coverage out to $|\eta| < 2.7$ as shown in Fig. \ref{muon_spectrometer}.
A toroid magnet system provides the magnetic field for momentum measurements, and is placed between layers of the MS.
Monitored Drift Tubes (MDT) are the most common tracking chambers in the MS, covering the full $\eta$ range.
The MDTs are composed of 3 cm diameter aluminum tubes filled with a gas mixture of 93\% Ar and 7\% CO$_2$, with a wire held at high voltage running through the tube.


\subsection{Changes for Run 3}\label{subsec:run3_config}